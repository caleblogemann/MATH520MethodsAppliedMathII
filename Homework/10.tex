\documentclass[11pt, oneside]{article}
\usepackage[letterpaper, margin=2cm]{geometry}
\usepackage{MATH520}

\begin{document}
\noindent \textbf{\Large{Caleb Logemann \\
MATH 520 Methods of Applied Math II \\
Homework 10
}}

\subsection*{Section 16.8}
\begin{enumerate}
  \item[\#2]
    Consider the Sturm-Liouville problem
    \[
      u'' + \lambda u = 0 \qquad 0 < x < 1
    \]
    \[
      u'(0) = u(1) = 0
    \]
    whose eigenvalues are the critical points of
    \[
      J(u) = \frac{\dintt{0}{1}{u'(x)}{x}}{\dintt{0}{1}{u(x)^2}{x}}
    \]
    on the space $H = \set{u \in H^1(0, 1): u(1) = 0}$.
    Use the Rayleigh-Ritz method to estimate the first two eigenvalues, and
    compare to the exact values.
    For best results, choose polynomial trial functions which resemble what the
    first two eigenfunctions should look like.

  \item[\#3]
    Use the result of Exercise 14 in Chapter 14 to give an alternate derivation
    of the fact the Dirichlet quotient achieves its minimum at $\psi_1$.
    (Hint: For $u \in H_0^1(\Omega)$ compute $\norm[H_0^1(\Omega)]{u}^2$ and
    $\norm[L^2(\Omega)]{u}^2$ by expanding in the eigenfunction basis.)

  \item[\#5]
    Let $A$ be an $m \times n$ real matrix, $b \in \RR^m$ and define
    $J(x) = \norm[2]{Ax - b}$ for $x \in \RR^n$.
    (Here $\norm[2]{x}$ denotes the 2 norm, the usual Euclidean distance on
    $\RR^m$)
    \begin{enumerate}
      \item[(a)]
        What is the Euler-Lagrange equation for the problem of minimizing J?

      \item[(b)]
        Under what circumstances does the Euler-Lagrange equation have a unique
        solution?

      \item[(c)]
        Under the circumstances will the solution of the Euler-Lagrange equation
        also be a solution of $Ax = b$.
    \end{enumerate}

  \item[\#13]
    Let $\Omega \subset \RR^N$ be a bounded open set,
    $\rho \in C(\overline{\Omega})$, $\rho(x) > 0$ in $\Omega$, and
    \[
      J(u) = \frac{\dintt{\Omega}{}{\norm{\nabla u(x)}^2}{x}}{\dintt{\Omega}{}{\rho(x)u(x)^2}{x}}
    \]
    \begin{enumerate}
      \item[(a)]
        Show that any nonzero critical point $u \in H_0^1(\Omega)$ of $J$ is a
        solution of the eigenvalue problem
        \[
          -\Delta u = \lambda \rho(x) u \qquad x \in \Omega
        \]
        \[
          u = 0 \qquad x \in \partial \Omega
        \]

      \item[(b)]
        Show that the eigenvalues are positive.

        \begin{proof}
          
        \end{proof}

      \item[(c)]
        If $\rho(x) \ge 1$ in $\Omega$ and $\lambda_1$ denotes the smallest
        eigenvalue, show that $\lambda_1 < \lambda_1^*$ where $\lambda_1^*$ is
        the corresponding first eigenvalue of $-\Delta$ in $\Omega$.

    \end{enumerate}

  \item[\#14]
    Define
    \[
      J(u) = \frac{1}{2} \dintt{\Omega}{}{(\Delta u)^2}{x} + \dintt{\Omega}{}{fu}{x}
    \]
    What PDE problem is satisfied by a critical point of $J$ over
    $\chi = H^2(\Omega) \cap H_0^1(\Omega)$?
    Make sure to specify any relevant boundary conditions.
    What is different if instead we let $\chi = H^2_0(\Omega)$?
\end{enumerate}
\end{document}
