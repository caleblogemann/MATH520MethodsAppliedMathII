\documentclass[11pt, oneside]{article}
\usepackage[letterpaper, margin=2cm]{geometry}
\usepackage{MATH520}

\begin{document}
\noindent \textbf{\Large{Caleb Logemann \\
MATH 520 Methods of Applied Math II \\
Homework 1
}}

\subsection*{Section 10.9}
\begin{enumerate}
  \item[\#3]
    Prove Proposition 10.1.
    Proposition 10.1 states that if $T$ is bounded on its domain then it has a
    unique norm preserving extension to $\overline{D(T)}$.
    That is to say there exists a unique linear operator
    $S:\overline{D(T)} \subset X \to Y$ such that $Sx = Tx$ for $x \in D(T)$ and
    $\norm{S} = \norm{T}$.

    \begin{proof}
      Let $S:\overline{D(T)} \subset X \to Y$ be defined as follows.
      \[
        Sx = \lim[n \to \infty]{Tx_n}
      \]
      where the sequence $\set{x_n}_{n = 1}^{\infty}$ is any sequence in $D(T)$
      that converges to $x$.
      Note that for any $x \in \overline{D(T)}$, $x$ is a limit point of $D(T)$
      so the sequence $\set{x_n}$ exists.
      Also since $T$ is bounded it is also continuous, so the limit always
      exists.

      Next I will show that $S$ is linear.
      Consider $x_1, x_2 \in \overline{D(T)}$ and $c_1, c_2 \in \CC$.
      Then there exists sequences in $D(T)$, $\set{a_n}_{n = 1}^{\infty}$ that
      converges to $x_1$ and $\set{b_n}_{n = 1}^{\infty}$ that converges to
      $x_2$.
      Now note that the sequence ${c_1 a_n + c_2 b_n}_{n = 1}^{\infty}$
      converges to $c_1 x_1 + c_2 x_2$ by the linearity of limits.
      Therefore
      \begin{align*}
        S(c_1 x_1 + c_2 x_2) &= \lim[n \to \infty]{T(c_1 a_n + c_2 b_n)}
        \intertext{Because $T$ is linear}
        S(c_1 x_1 + c_2 x_2) &= \lim[n \to \infty]{c_1 T(a_n) + c_2 T(b_n)}
        \intertext{By the linearity of limits}
        S(c_1 x_1 + c_2 x_2) &= c_1 \lim[n \to \infty]{T(a_n)} + c_2 \lim[n \to \infty]{T(b_n)} \\
        S(c_1 x_1 + c_2 x_2) &= c_1 S(x_1) + c_2 S(x_2)
      \end{align*}
      This shows that $S$ is a linear operator.

      Next I will show that $Sx = Tx$ for $x \in D(T)$.
      Let $\set{x_n}_{n = 1}^{\infty}$ converge to $x$ in $D(T)$, then because
      $T$ is continuous, $\lim[n \to \infty]{T(x_n)} = T(x)$.
      Therefore $Sx = Tx$.

      Lastly I will show that $\norm{S} = \norm{T}$.
      Consider the following.
      \begin{align*}
        \norm{S} &= \sup*[x \in \overline{D(T)}]{\frac{\norm[Y]{Sx}}{\norm[X]{x}} \\
      \end{align*}
    \end{proof}

  \pagebreak
  \item[\#6]
    Show that a linear operator $T:\CC^N \to \CC^M$ is always bounded for
    any choice of norms on $\CC^N$ and $\CC^M$.

    \begin{proof}
      Let $T:\CC^N \to \CC^M$ be a linear operator.
      It is known that any linear operator from $\CC^N \to \CC^M$ can be
      expressed as a matrix multiplication, that is there exists matrix
      $A \in \CC^{m \times n}$ such that $Tx = Ax$ for every $x \in \CC^N$.
      It is well known that for finite dimensional vector spaces any two norms
      are equivalently.
      More precisely let $\norm[1]{\cdot}$ and $\norm[2]{\cdot}$ be norms on a
      finite dimensional vector space, then there exists constants $C_1$ and $C_2$ such
      that
      \[
        0 < C_1 \le \frac{\norm[1]{x}}{\norm[2]{x}} \le C_2 < \infty
      \]
      for any nonzero $x$ in the vector space.
      Since both $N$ and $M$ are finite, norms on $\CC^N$ are equivalent
      and norms on $\CC^M$ are equivalent.
      Therefore I will let $\norm[N]{\cdot}$ represent any norm on $\CC^N$ and
      $\norm[M]{\cdot}$ represent any norm on $\CC^M$.
    \end{proof}

  \pagebreak
  \item[\#7]
    If $T, T^{-1} \in \mcB(\v{H})$ show that $\p{T^*}^{-1} \in \mcB(\v{H}$ and
    $\p{T^*}^{-1} = \p{T^{-1}}^*$.

    \begin{proof}
      Let $\v{H}$ be a Hilbert Space and let $T \in \mcB(\v{H})$ with 
    \end{proof}

  \pagebreak
  \item[\#14]
    If $T \in \mcB(\v{H})$ show that $T^*$ restricted to $R(T)$ is one-to-one.

    \begin{proof}
      Let $\v{H}$ be a Hilbert space and let $T \in \mcB(\v{H})$.
      Consider $T^*$ restricted to $R(T)$.
      In other words let $S = T^*$ where $D(S) = R(T)$.
    \end{proof}
\end{enumerate}
\end{document}
