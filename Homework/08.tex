\documentclass[11pt, oneside]{article}
\usepackage[letterpaper, margin=2cm]{geometry}
\usepackage{MATH520}

\begin{document}
\noindent \textbf{\Large{Caleb Logemann \\
MATH 520 Methods of Applied Math II \\
Homework 7
}}

\subsection*{Section 13.6}
\begin{enumerate}
  \item[\#7]
    Let $\set{\lambda_j}_{j = 1}^{\infty}$ be a sequence of nonzero real
    numbers satisfying
    \[
      \sum{j = 1}{\infty}{\lambda_j^2} < \infty
    \]
    Construct a summetric Hilber Schmidt kernel $K$ such that the corresonding
    integral operator has eigenvalues $\lambda_j$, $j = 1, 2, \ldots$ and for
    which $0$ is an eigenvalue of infinite multiplicity.
    ( Suggestion: look for such a $K$ in the form
    $K(x, y) = \sum{j = 1}{\infty}{\lambda_j u_j(x) \overline{u_j(y)}}$ where
    $\set{u_j}$ are orthonormal but not complete in $L^2(\Omega)$.)

  \pagebreak
  \item[\#12]
    Compute the singular value decomposition of the Volterra operator
    \[
      Tu(x) = \dintt{0}{x}{u(s)}{s}
    \]
    in $L^2(0, 1)$ and use it to find $\norm{T}$.
    Is $T$ normal?
    (Suggestion: The equation $T^* T u = \lambda u$ is equivalent to an ODE
    eigenvalue problem which you can solve explicitly.)

    In order to compute the singular value decomposition of the Volterra
    operator, we must first find the operator $T^*$, so that we may construct
    $S = T^* T$.
    Since the Volterra operator is an integral operator in may be rewritten as
    \[
      Tu(x) = \dintt{0}{1}{K(x, y)u(y)}{y}
    \]
    where the kernel $K(x, y)$ is
    \[
      K(x, y) =
      \begin{cases}
        1 & y < x \\
        0 & y > x
      \end{cases}.
    \]
    The adjoint of any integral operator is another integral operator with
    kernel, $\overline{K(y, x)}$.
    In this case
    \[
      \overline{K(y, x)} =
      \begin{cases}
        1 & x < y \\
        0 & x > y
      \end{cases}.
    \]
    Therefore
    \begin{align*}
      T^* u(x) &= \dintt{0}{1}{\overline{K(y, x)}u(y)}{y} \\
      &= \dintt{x}{1}{u(y)}{y}.
    \end{align*}

    Now I will define $S = T^* T$ which can be expressed as
    \begin{align*}
      Su(x) &= T^* T u(x) \\
      &= T^* \dintt{0}{x}{u(s)}{s} \\
      &= \dintt{x}{1}{\dintt{0}{y}{u(s)}{s}}{y}
    \end{align*}

    In order to compute the singular value decomposition of $T$ we must first
    find the eigenvalues and eigenfunctions of $S$.
    This means solving $Su = \lambda u$.
    This integral equation can be changed into a differential equation as
    follows.
    \begin{align*}
      \lambda u(x) &= Su(x) \\
      \lambda u(x) &= \dintt{x}{1}{\dintt{0}{y}{u(s)}{s}}{y} \\
      \lambda u(x) &= -\dintt{1}{x}{\dintt{0}{y}{u(s)}{s}}{y} \\
      \lambda u'(x) &= -\dintt{0}{x}{u(s)}{s} \\
      \lambda u''(x) &= -u(x) \\
    \end{align*}
    The boundary conditions for this differential equation can be found by
    evaluating $Su(x)$ at $x = 1$.
    Clearly $Su(1) = 0$ so this implies that $u(1) = 0$.
    The other boundary condition can be found by evaluating $\da{Su(x)}{x}$
    at $x = 0$.
    In this case the result is $u'(0) = 0$.

    So now the original integral equation is identical to the following
    differential equation.
    \begin{align*}
      \lambda u''(x) + u(x) = 0 \\
      u(1) = 0 \qquad u'(0) = 0
    \end{align*}
    This differntial equation can be solved using the characteristic polynomial
    \[
      \lambda r^2 + 1 = 0
    \]
    The roots of this polynomial are $r = \sqrt{1/\lambda} i$ and
    $r = -\sqrt{1/\lambda} i$.
    Since these are complex roots the solutions will be of the form
    \[
      u(x) = c_1 \sin{\sqrt{1/\lambda} x} + c_2 \cos{\sqrt{1/\lambda} x}
    \]
    Applying the boundary conditions can find the appropriate constants
    \begin{align*}
      u'(0) &= c_1 \sqrt{1/\lambda} \cos{\sqrt{1/\lambda} 0} - c_2 \sqrt{1/\lambda} \sin{\sqrt{1/\lambda} 0} \\
      0 &= c_1 \sqrt{1/\lambda} \\
      0 &= c_1
      u(1) &= c_2 \cos{\sqrt{1/\lambda}}
    \end{align*}
\end{enumerate}

\pagebreak
\subsection*{Section 14.5}
\begin{enumerate}
  \item[\#1]
    Let $Lu = (x - 2)u'' + (1 - x)u' + u$ on $(0, 1)$.
    \begin{enumerate}
      \item[(a)]
        Find the Green's function for
        \[
          Lu = f \qquad u'(0) = 0 \quad u(1) = 0
        \]
        (Hint First show that $x - 1$, $e^x$ are linearly independent
        solutions of $Lu = 0$.

      \item[(b)]
        Find the adjoint operator and boundary conditions.
    \end{enumerate}

  \pagebreak
  \item[\#2]
    Let
    \[
      Tu = - \d*{x \d{u}{x}}{x}
    \]
    on the domain
    \[
      D(T) = \set{u \in H^2(1, 2): u(1) = u(2) = 0}
    \]
    \begin{enumerate}
      \item[(a)]
        Show that $N(T) = \set{0}$.
      \item[(b)]
        Find the Green's function for the boundary value problem $Tu = f$.
      \item[(c)]
        State and prove a result about the continuous dependence of the
        solution $u$ on $f$ in part (b).
    \end{enumerate}

  \pagebreak
  \item[\#4]
    Prove the validity of (14.1.22).
    (Suggestions: start by writing $u(x)$ in the form
    \[
      u(x) = \phi_2(x) \dintt{a}{x}{C_2(y)f(y)}{y} + \phi_1(x) \dintt{x}{b}{C_1(y)f(y)}{y}
    \]
    and note that some of the terms that arise in the expression for u'(x) will
    cancel.)
\end{enumerate}
\end{document}
