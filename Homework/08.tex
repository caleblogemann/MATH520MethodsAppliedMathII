\documentclass[11pt, oneside]{article}
\usepackage[letterpaper, margin=2cm]{geometry}
\usepackage{MATH520}

\begin{document}
\noindent \textbf{\Large{Caleb Logemann \\
MATH 520 Methods of Applied Math II \\
Homework 7
}}

\subsection*{Section 13.6}
\begin{enumerate}
  \item[\#7]
    Let $\set{\lambda_j}_{j = 1}^{\infty}$ be a sequence of nonzero real
    numbers satisfying
    \[
      \sum{j = 1}{\infty}{\lambda_j^2} < \infty
    \]
    Construct a symmetric Hilbert Schmidt kernel $K$ such that the corresonding
    integral operator has eigenvalues $\lambda_j$, $j = 1, 2, \ldots$ and for
    which $0$ is an eigenvalue of infinite multiplicity.
    ( Suggestion: look for such a $K$ in the form
    $K(x, y) = \sum{j = 1}{\infty}{\lambda_j u_j(x) \overline{u_j(y)}}$ where
    $\set{u_j}$ are orthonormal but not complete in $L^2(\Omega)$.)

    Let $u_j = e^{1/\lambda_j x}$ and let
    $K(x, y) = \sum{j = 1}{\infty}{\lambda_j u_j(x) \overline{u_j(y)}}$.

  \pagebreak
  \item[\#12] % Done
    Compute the singular value decomposition of the Volterra operator
    \[
      Tu(x) = \dintt{0}{x}{u(s)}{s}
    \]
    in $L^2(0, 1)$ and use it to find $\norm{T}$.
    Is $T$ normal?
    (Suggestion: The equation $T^* T u = \lambda u$ is equivalent to an ODE
    eigenvalue problem which you can solve explicitly.)

    In order to compute the singular value decomposition of the Volterra
    operator, we must first find the operator $T^*$, so that we may construct
    $S = T^* T$.
    Since the Volterra operator is an integral operator in may be rewritten as
    \[
      Tu(x) = \dintt{0}{1}{K(x, y)u(y)}{y}
    \]
    where the kernel $K(x, y)$ is
    \[
      K(x, y) =
      \begin{cases}
        1 & y < x \\
        0 & y > x
      \end{cases}.
    \]
    The adjoint of any integral operator is another integral operator with
    kernel, $\overline{K(y, x)}$.
    In this case
    \[
      \overline{K(y, x)} =
      \begin{cases}
        1 & x < y \\
        0 & x > y
      \end{cases}.
    \]
    Therefore
    \begin{align*}
      T^* u(x) &= \dintt{0}{1}{\overline{K(y, x)}u(y)}{y} \\
      &= \dintt{x}{1}{u(y)}{y}.
    \end{align*}

    Now I will define $S = T^* T$ which can be expressed as
    \begin{align*}
      Su(x) &= T^* T u(x) \\
      &= T^* \dintt{0}{x}{u(s)}{s} \\
      &= \dintt{x}{1}{\dintt{0}{y}{u(s)}{s}}{y}
    \end{align*}

    In order to compute the singular value decomposition of $T$ we must first
    find the eigenvalues and eigenfunctions of $S$.
    This means solving $Su = \lambda u$.
    Note that we have previously shown that $\lambda \ge 0$ because $S = T^* T$.
    Also note that if $\lambda = 0$, then
    \[
      \dintt{x}{1}{\dintt{0}{y}{u(s)}{s}}{y} = 0
    \]
    which implies that $u = 0$.
    Therefore 0 cannot be an eigenvalue, and also $\lambda > 0$.
    In order to find the eigenvalues, this integral equation can be changed into
    a differential equation as follows.
    \begin{align*}
      \lambda u(x) &= Su(x) \\
      \lambda u(x) &= \dintt{x}{1}{\dintt{0}{y}{u(s)}{s}}{y} \\
      \lambda u(x) &= -\dintt{1}{x}{\dintt{0}{y}{u(s)}{s}}{y} \\
      \lambda u'(x) &= -\dintt{0}{x}{u(s)}{s} \\
      \lambda u''(x) &= -u(x) \\
    \end{align*}
    The boundary conditions for this differential equation can be found by
    evaluating $Su(x)$ at $x = 1$.
    Clearly $Su(1) = 0$ so this implies that $u(1) = 0$.
    The other boundary condition can be found by evaluating $\da{Su(x)}{x}$
    at $x = 0$.
    In this case the result is $u'(0) = 0$.

    So now the original integral equation is identical to the following
    differential equation.
    \begin{align*}
      \lambda u''(x) + u(x) = 0 \\
      u(1) = 0 \qquad u'(0) = 0
    \end{align*}
    This differntial equation can be solved using the characteristic polynomial
    \[
      \lambda r^2 + 1 = 0
    \]
    The roots of this polynomial are $r = \sqrt{1/\lambda} i$ and
    $r = -\sqrt{1/\lambda} i$.
    Since these are complex roots the solutions will be of the form
    \[
      u(x) = c_1 \sin{\sqrt{1/\lambda} x} + c_2 \cos{\sqrt{1/\lambda} x}
    \]
    Applying the boundary conditions can find the appropriate constants
    \begin{align*}
      u'(0) &= c_1 \sqrt{1/\lambda} \cos{\sqrt{1/\lambda} 0} - c_2 \sqrt{1/\lambda} \sin{\sqrt{1/\lambda} 0} \\
      0 &= c_1 \sqrt{1/\lambda} \\
      0 &= c_1 \\
      u(1) &= c_2 \cos{\sqrt{1/\lambda}} \\
      0 &= c_2 \cos{\sqrt{1/\lambda}}
    \end{align*}
    Since $c_2$ can not equal zero, otherwise $u = 0$, this implies that
    $\cos{\sqrt{1/\lambda}} = 0$.
    This is equivalent to $\sqrt{1/\lambda} = \pi/2 + k\pi$, or the eigenvalues
    are $\lambda_k = \frac{1}{\p{\pi(1/2 + k)}^2}$, which can be indexed by
    $k \ge 0$.
    The eigenfunctions then are $u_k(x) = c \cos{\sqrt{1/\lambda_k} x}$.
    We wish to normalize these eigenfunctions in order to create the singular
    value decomposition of $T$.
    \begin{align*}
      \norm{u_k}^2 &= \dintt{0}{1}{c^2 \cos{\frac{1}{\sqrt{\lambda_k}} x}^2}{x} \\
      1 &= c^2 \dintt{0}{1}{\frac{1 + \cos{\frac{2}{\sqrt{\lambda_k}} x}}{2}}{x} \\
      1 &= c^2 \p{\frac{1}{2} + \frac{\sqrt{\lambda_k}}{4} \eval{\sin{\frac{2}{\sqrt{\lambda_k}} x}}{x = 0}{1}} \\
      1 &= c^2 \p{\frac{1}{2} + \frac{\sqrt{\lambda_k}}{4} \sin{\frac{2}{\sqrt{\lambda_k}}}} \\
      1 &= c^2 \p{\frac{1}{2} + \frac{\sqrt{\lambda_k}}{4} \sin{\pi + 2k\pi}} \\
      1 &= c^2 \frac{1}{2} \\
      2 &= c^2 \\
      c &= \sqrt{2}
    \end{align*}
    The eigenfunctions are thus $u_k(x) = \sqrt{2} \cos{\frac{1}{\sqrt{\lambda_k}} x}$.
    By Theorem 13.10 these eigenfunctions form an orthonormal basis of $L^2(0, 1)$.

    The singular values are the square roots of the eigenvalues, that is
    $\sigma_k = \sqrt{\lambda_k} = \frac{1}{\pi(1/2 + k)}$.

    Lastly we can compute $v_k = Tu_k/\sigma_k$.
    \begin{align*}
      v_k &= \frac{1}{\sigma_k} Tu_k \\
      &= \frac{1}{\sigma_k} \dintt{0}{x}{u(y)}{y} \\
      &= \frac{1}{\sigma_k} \dintt{0}{x}{\sqrt{2} \cos{\frac{1}{\sqrt{\lambda_k}} y}}{y} \\
      &= \frac{\sqrt{2\lambda_k}}{\sigma_k} \eval{\sin{\frac{1}{\sqrt{\lambda_k}} y}}{y=0}{x} \\
      &= \sqrt{2} \sin{\frac{1}{\sqrt{\lambda_k}} x} \\
    \end{align*}

    Now that we have found $\sigma_k$, $u_k$, and $v_k$, the singular value
    decomposition of $T$ is $\sum{n = 0}{\infty}{\sigma_n \abr{u, u_n} v_n}$.

    The norm of $T$ is the largest singular value, that is
    $\norm{T} = \sigma_0 = \frac{2}{\pi}$.
    Also $T$ is not normal because if $T$ was normal then $Tu_n = \sigma_n u_n$
    or $v_n = u_n$.
    However this is not the case so $T$ is not normal.
\end{enumerate}

\pagebreak
\subsection*{Section 14.5}
\begin{enumerate}
  \item[\#1] % Done
    Let $Lu = (x - 2)u'' + (1 - x)u' + u$ on $(0, 1)$.
    \begin{enumerate}
      \item[(a)] % Done
        Find the Green's function for
        \[
          Lu = f \qquad u'(0) = 0 \quad u(1) = 0
        \]
        (Hint First show that $x - 1$, $e^x$ are linearly independent
        solutions of $Lu = 0$.)

        First consider $Lu = 0$.
        Let $u(x) = x - 1$, and consider $Lu$.
        \begin{align*}
          Lu(x) &= (x - 2)u''(x) + (1 - x)u'(x) + u(x) \\
          &= (x - 2)0 + (1 - x)1 + (x - 1) \\
          &= 0
        \end{align*}
        Let $u(x) = e^x$, then
        \begin{align*}
          Lu(x) &= (x - 2)e^x + (1 - x)e^x + e^x \\
          &= (x - 2 + 1 - x + 1)e^x \\
          &= 0
        \end{align*}
        Also these two functions are linearly independent because they are not
        multiples of one another.

        The boundary conditions for this problem are $B_1 u = u'(0) = 0$ and
        $B_2 u = u(1) = 0$.
        Let $\phi_1(x) = e^x - (x - 1)$ and $\phi_2(x) = x - 1$ and note that
        $L\phi_1 = 0$, $B_1 \phi_1 = 0$, $L\phi_2 = 0$ and $B_2 \phi_2 = 0$.
        Thus we can use $\phi_1$ and $\phi_2$ to build the Green's function for
        this differential equation.

        The next thing to do is compute $C_1(y)$ and $C_2(y)$, which first
        requires computing $W(y)$, the Wronskian of $\phi_1$ and $\phi_2$.
        \begin{align*}
          W(x) &=
          \begin{vmatrix}
            e^x - (x - 1) & x - 1 \\
            e^x - 1 & 1
          \end{vmatrix} \\
          &= e^x - (x - 1) - (x - 1)(e^x - 1) \\
          &= (2 - x)e^x
        \end{align*}

        Now we can compute $C_1$ and $C_2$ as
        \begin{align*}
          C_1(y) &= \frac{\phi_1(y)}{a_2(y) W(y)} \\
          &= \frac{y - 1}{(y - 2)(2 - y)e^y} \\
          &= \frac{1 - y}{(y - 2)^2 e^y} \\
          C_2(y) &= \frac{\phi_2(y)}{a_2(y) W(y)} \\
          &= \frac{e^y - (y - 1)}{(y - 2)(2 - y)e^y} \\
          &= \frac{y - 1 - e^y}{(y - 2)^2 e^y} \\
        \end{align*}

        Finally the Green's Function for this differential equation is
        \begin{align*}
          G(x, y) &=
          \begin{cases}
            C_1(y) \phi_1(x) & 0 < x < y < 1 \\
            C_2(y) \phi_2(y) & 0 < y < x < 1
          \end{cases} \\
          &=
          \begin{cases}
            \frac{1 - y}{(y - 2)^2 e^y} \p{e^x - (x - 1)} & 0 < x < y < 1 \\
            \frac{y - 1 - e^y}{(y - 2)^2 e^y} (x - 1) & 0 < y < x < 1
          \end{cases}
        \end{align*}

      \item[(b)]
        Find the adjoint operator and boundary conditions.

        The adjoint operator is given by
        \[
          L^* u = (\overline{a_2} u)'' - (\overline{a_1}u)' + \overline{a_0}u
        \]
        Therefore
        \[
          L^* u = (x - 2)u'' + (3 - x)u'
        \]

        The boundary functional for the adjoint is
        \[
          J(\phi, \psi) = (x - 2)(\phi' \overline{\psi} - \phi \overline{\psi}') - x \phi \overline{\psi}
        \]
        where $\phi \in D(T)$ and $\psi \in D(T^*)$.
        The boundary conditions are equivalent to $\eval{J(\phi, \psi)}{x = 0}{1} = 0$.
        \[
          -1(\phi'(1) \overline{\psi(1)} - \phi(1) \overline{\psi}'(1)) - \phi(1) \overline{\psi(1)} + 2(\phi'(0) \overline{\psi(0)} - \phi(0) \overline{\psi(0)}') = 0
        \]
        Since $\phi'(0) = 0$ and $\phi(1) = 0$, this is equivalent to
        \[
          -\phi'(1) \overline{\psi(1)} - 2\phi(0) \overline{\psi(0)}') = 0
        \]
        Now since $\phi'(1)$ and $\phi(0)$ can be anything the two boundary conditions are
        \begin{align*}
          B_1 \psi = -2\psi'(0) = 0 \\
          B_2 \psi = -\psi(1) = 0 \\
        \end{align*}
    \end{enumerate}

  \pagebreak
  \item[\#2] % Done
    Let
    \[
      Tu = - \d*{x \d{u}{x}}{x}
    \]
    on the domain
    \[
      D(T) = \set{u \in H^2(1, 2): u(1) = u(2) = 0}
    \]
    \begin{enumerate}
      \item[(a)] % Done
        Show that $N(T) = \set{0}$.

        \begin{proof}
          First note that clearly $0 \in N(T)$.
          Now consider $u \neq 0$ such that $u \in N(T)$, then
          \begin{align*}
            Tu &= 0 \\
            - \d*{x \d{u}{x}}{x} &= 0 \\
            \d*{x \d{u}{x}}{x} &= 0 \\
            x \d{u}{x} &= c_1 \\
            \d{u}{x} &= \frac{c_1}{x} \\
            u(x) &= \intt{\frac{c_1}{x}}{x} + c_2\\
            u(x) &= c_1 \ln{x} + c_2\\
          \end{align*}
          If $u \in D(T)$, then $u$ must satisfy $u(1) = u(2) = 0$.
          \begin{align*}
            u(1) &= c_1 \ln{1} + c_2 \\
            0 &= c_2 \\
            u(2) &= c_1 \ln{2} \\
            0 &= c_1
          \end{align*}
          This shows that $u = 0$, which contradicts that $u \neq 0$.
          This shows that $N(T) = \set{0}$.
        \end{proof}

      \item[(b)] % Done
        Find the Green's function for the boundary value problem $Tu = f$.

        In order to find the Green's function for the boundary value problem,
        we must first find $\phi_1$ and $\phi_2$ where $L\phi_1 = 0$,
        $B_1 \phi_1 = 0$, $L\phi_2 = 0$, and $B_2 \phi_2 = 0$.
        For this problem $Lu = - \d*{x \d{u}{x}}{x}$, $B_1 u = u(1)$, and $B_2 u = u(2)$.
        We have already shown in part (a) that any function that satisfies $Lu = 0$ is
        of the form $u(x) = c_1 \ln{x} + c_2$.
        Therefore let $\phi_1(x) = c_1 \ln{x} + c_2$ and let $B_1 \phi_1 = 0$,
        then $\phi_1(1) = c_1 \ln{1} + c_2 = 0$ or $c_2 = 0$.
        Also let $c_1 = 1$, then $\phi_1(x) = \ln{x}$.
        Now $\phi_2(x)$ is of the form $c_1 \ln{x} + c_2$.
        If $\phi_2(2) = 0$, then $c_1 \ln{2} + c_2 = 0$ and $c_2 = -c_1 \ln{2}$.
        Then let $c_1 = 1$ and $\phi_2(x) = \ln{x} - \ln{2}$.

        The next step is to compute the Wronskian.
        \begin{align*}
          W(y) &=
          \begin{vmatrix}
            \ln{y} & \ln{y} - \ln{2} \\
            1/y & 1/y
          \end{vmatrix} \\
          &= \frac{\ln{y}}{y} - \frac{\ln{y} - \ln{2}}{y} \\
          &= \frac{\ln{2}}{y}
        \end{align*}
        We also need to find $a_2(x)$, which can be found using the product rule
        \begin{align*}
          -\d*{x \d{u}{x}}{x} &= - x \d[2]{u}{x} - \d{u}{x}
        \end{align*}
        Therefore $a_2(x) = -x$.

        Now $C_1(y)$ and $C_2(y)$ are
        \begin{align*}
          C_1(y) &= \frac{\phi_2(y)}{a_2(y)W(y)} \\
          &= \frac{\ln{y} - \ln{2}}{-y\p{\ln{2}/y}} \\
          &= \frac{\ln{2} - \ln{y}}{\ln{2}} \\
          C_2(y) &= \frac{\phi_1(y)}{a_2(y)W(y)} \\
          &= \frac{\ln{y}}{-y\p{\ln{2}/y}} \\
          &= -\frac{\ln{y}}{\ln{2}}
        \end{align*}

        Finally the Green's function is
        \begin{align*}
          G(x, y) &=
          \begin{cases}
            C_1(y) \phi_1(x) & 0 < x < y < 1 \\
            C_2(y) \phi_2(x) & 0 < y < x < 1
          \end{cases} \\
          &=
          \begin{cases}
            \frac{\ln{2} - \ln{y}}{\ln{2}}\ln{y} & 0 < x < y < 1 \\
            -\frac{\ln{y}}{\ln{2}} \p{\ln{y} - \ln{2}} & 0 < y < x < 1
          \end{cases}
        \end{align*}

      \item[(c)] % Done
        State and prove a result about the continuous dependence of the
        solution $u$ on $f$ in part (b).

        Let $f_1, f_2 \in R(T)$, then there exists $u_1, u_2 \in D(T)$ such that
        $Tu_1 = f_1$ and $Tu_2 = f_2$ or equivalently $u_1 = Sf_1$
        and $u_2 = Sf_2$, where $Sf = \dintt{1}{2}{G(x, y)f(y)}{y}$.
        I will define $\Delta f = f_1 - f_2$ and $\Delta u = u_1 - u_2$ and since
        $S$ is linear this implies that $\Delta u = S \Delta f$.
        The magnitude of $\Delta u$ satisfies
        \[
          \norm{\Delta u} \le M \norm{\Delta f}
        \]
        where $M > 0$.

        \begin{proof}
          This can be proved by noting that $G(x, y)$ is bounded on
          $\br{1, 2} \times \br{1, 2}$.
          Since $G(x, y)$ is bounded this implies that the integral operator
          $S$ is bounded, and therefore
          \[
            \norm{\Delta u} = \norm{S\Delta f} \le \norm{S} \norm{\Delta f}.
          \]
        \end{proof}
    \end{enumerate}

  \pagebreak
  \item[\#4] % Done
    Prove the validity of (14.1.22).
    (Suggestions: start by writing $u(x)$ in the form
    \[
      u(x) = \phi_2(x) \dintt{a}{x}{C_2(y)f(y)}{y} + \phi_1(x) \dintt{x}{b}{C_1(y)f(y)}{y}
    \]
    and note that some of the terms that arise in the expression for u'(x) will
    cancel.)

    \begin{proof}
      First let $u(x) = Sf = \dintt{a}{b}{G(x, y)f(y)}{y}$, and then I will show that
      $u$ satisfies $Lu = f$.
      Using the piecewise definition of $G(x, y)$, it is possible to rewrite $u$
      as
      \[
        u(x) = \phi_2(x) \dintt{a}{x}{C_2(y)f(y)}{y} + \phi_1(x) \dintt{x}{b}{C_1(y)f(y)}{y}
      \]
      In order to compute $Lu$, we must first compute $u'$ and $u''$.
      For simplicity $I_2(x) = \dintt{a}{x}{C_2(y)f(y)}{y}$ and
      $I_1 = \dintt{x}{b}{C_1(y)f(y)}{y}$.
      \begin{align*}
        u'(x) &= \phi_2'(x) I_2(x) + \phi_2(x) C_2(x) f(x) + \phi_1'(x) I_1(x) - \phi_1(x) C_1(x) f(x)
        \intertext{Since $\phi_2(X) C_2(x) = \phi_1(x) C_1(x)$}
        u'(x) &= \phi_2'(x) I_2(x) + \phi_1'(x) I_1(x) \\
        u''(x) &= \phi_2''(x) I_2(x) + \phi'_2(x) C_2(x) f(x) + \phi_1''(x) I_1(x) - \phi_1'(x)C_1(x)f(x)
      \end{align*}
      Now $Lu$ can be found.
      \begin{align*}
        Lu(x) &= a_2(x) u''(x) + a_1 u'(x) + a_0 u(x) \\
        &= a_2(x) \p{\phi_2''(x) I_2(x) + \phi'_2(x) C_2(x) f(x) + \phi_1''(x) I_1(x) - \phi_1'(x)C_1(x)f(x)} \\
        &+ a_1(x) \p{\phi_2'(x) I_2(x) + \phi_1'(x) I_1(x)} + a_0 \p{\phi_2(x) I_2(x) + \phi_1(x) I_1(x)} \\
        &= I_2(x) \p{a_2(x) \phi_2''(x) + a_1(x) \phi_2'(x) + a_0 \phi_2(x)} \\
        &+ I_1(x) \p{a_2(x) \phi_1''(x) + a_1(x) \phi_1'(x) + a_0(x) \phi_1(x)} + \phi_2'(x) C_2(x) f(x) - \phi_1'(x)C_1(x)f(x) \\
        &= I_2(x) L\phi_2(x) + I_1(x) L\phi_1(x) + a_2(x)\phi_2'(x) C_2(x) f(x) - a_2(x)\phi_1'(x)C_1(x)f(x)
        \intertext{Since $L\phi_2(x) = 0$ and $L\phi_1(x) = 0$.}
        &= a_2(x)\phi_2'(x) C_2(x) f(x) - a_2(x)\phi_1'(x)C_1(x)f(x) \\
        &= a_2(x)\p{\frac{\phi_2'(x)\phi_1(x) - \phi_1'(x)\phi_2(x)}{a_2(x) W(x)}} f(x) \\
        &= \p{\frac{\phi_2'(x)\phi_1(x) - \phi_1'(x)\phi_2(x)}{W(x)}} f(x) \\
        &= f(x)
      \end{align*}
      This shows that $Lu = f$.
      Finally I will show that $B_1u = 0$ and $B_2 u = 0$.
      \begin{align*}
        B_1 u &= c_1 u(a) + c_2 u'(a) \\
        &= c_1 \p{\phi_2(a) I_2(a) + \phi_1(a) I_1(a)} + c_2 \p{\phi_2'(a) I_2(a) + \phi_1'(a) I_1(a)}
        \intertext{Since $I_2(a) = 0$}
        &= c_1 \phi_1(a) I_1(a) + c_2 \phi_1'(a) I_1(a) \\
        &= I_1(a) \p{c_1 \phi_1(a) + c_2 \phi_1'(a)} \\
        &= I_1(a) B_1 \phi_1
        \intertext{Since $B_1 \phi_1 = 0$}
        B_1 u &= 0 \\
        B_2 u &= c_3 u(b) + c_4 u'(b) \\
        &= c_3 \p{\phi_2(b) I_2(b) + \phi_1(b) I_1(b)} + c_4 \p{\phi_2'(b) I_2(b) + \phi_1'(b) I_1(b)}
        \intertext{Since $I_1(b) = 0$}
        &= c_3 \phi_2(b) I_2(b) + c_4 \phi_2'(b) I_2(b) \\
        &= I_2(b) \p{c_3 \phi_2(b) + c_4 \phi_2'(b)} \\
        &= I_2(b) B_2 \phi_2
        \intertext{Since $B_2 \phi_2 = 0$}
        B_2 u &= 0
      \end{align*}
      This shows that $u$ satisfies the boundary conditions and the differential equation.
    \end{proof}
\end{enumerate}
\end{document}
