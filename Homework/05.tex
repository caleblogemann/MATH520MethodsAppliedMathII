\documentclass[11pt, oneside]{article}
\usepackage[letterpaper, margin=2cm]{geometry}
\usepackage{MATH520}

\begin{document}
\noindent \textbf{\Large{Caleb Logemann \\
MATH 520 Methods of Applied Math II \\
Homework 5
}}

\subsection*{Section 12.4}
\begin{enumerate}
  \item[\#6] % Done
    Let $T$ denote the right shift operator on $\mcl^2$.
    \begin{enumerate}
      \item[(a)] % Done
        Show that $\sigma_p(T) = \varnothing$.

        \begin{proof}
          First I will let $S_+ = T$, so as to better represent the right shift
          operator.
          In order to show that $\sigma_p(S_+) = \varnothing$, we must show that
          $S_+$ has no eigenvalues.
          First I will show that $0$ is not an eigenvalue, if $0$ was an
          eigenvalue then $S_+ x = 0 x = 0$ would have a nonzero solution
          $x \in \mcl^2$.
          However the equation $S_+ x = 0$ guarantees that
          \[
            \p{S_+ x}_k = x_{k-1} = 0
          \]
          for $k \ge 1$, which implies that $x = 0$.
          This shows that the only solution to $S_+ x = 0$ is $x = 0$, so $0$
          is not an eigenvalue of $S_+$, e.g. $0 \not\in \sigma_p(S_+)$.

          Next I will show that no nonzero value can be an eigenvalue.
          Assume to the contrary that $\lambda \neq 0 \in \sigma_p(T)$, that is
          $\lambda$ is an eigenvalue of $T$.
          This implies that $S_+ x = \lambda x$ has a nonzero solution
          $x \in \mcl^2$.
          Both $S_+ x$ and $\lambda x$ are sequences and for the sequences to be
          equal each term in the sequences must be equal.
          Therefore I will compare the terms of these sequences.
          Note that by definition $\p{S_+ x}_k = x_{k-1}$ for $k \ge 1$ and
          $\p{S_+ x}_0 = 0$.
          Using this definition, in $S_+ x = \lambda x$ two conditions arise
          first for $k = 0$
          \[
            \lambda x_0 = \p{\lambda x}_0 = \p{S_+ x}_0 = 0
          \]
          and for $k \ge 1$
          \[
            \lambda x_k = \p{\lambda x}_k = \p{S_+ x}_k = x_{k-1}
          \]
          The first condition shows that $x_0 = 0$, and the second that
          $x_k = \frac{x_{k-1}}{\lambda}$.
          However these two statements together inductively show that $x_k = 0$
          for $k \ge 0$.
          Thus $x = 0$ is the only solution to the equation $S_+ x = \lambda x$,
          and $\lambda \not\in \sigma_p(S_+)$.
          This shows that no complex number can be an eigenvalue of $S_+$ and so
          $\sigma_p(S_+) = \varnothing$.
        \end{proof}

      \item[(b)] % Done
        Show that $\sigma_c(T) = \set{\lambda: \abs{\lambda}= 1}$.

      \item[(c)] % Done
        Show that $\sigma_r(T) = \set{\lambda: \abs{\lambda} < 1}$.

        \begin{proof}
          Again I will let $S_+ = T$ and I will prove (b) and (c) simultaneaously.
          From part (a) it is clear that for any $\lambda$, the operator
          $\lambda I - S_+$ is one-to-one.
          If $\lambda I - S_+$ was not one-to-one then
          $\lambda \in \sigma_p(S_+)$, however we have already shown that
          $\sigma_p(S_+)$ is empty.
          Thus for any $\lambda \in \CC$, $\lambda$ must be in the resolvant
          set, the continuous spectrum, or the residual spectrum.
          From Example 12.5 it is known that for $\abs{\lambda} > 1$, then
          $\lambda \in \rho(S_+)$.
          This is shown by noting that for bounded operators
          $\lambda \in \sigma(S_+)$ implies that
          $\abs{\lambda} \le \norm{S_+} = 1$.
          Let $\lambda \in \sigma(S_+)$, then $\abs{\lambda} \le 1$.
          Note that $I, S_+ \in \mcB(\mcl^2)$, so
          \[
            \p{\lambda I - S_+}^* = \overline{\lambda} I^* - S_+^* = \overline{\lambda} I - S_-.
          \]
          Also since $\lambda I - S_+$ is densely defined linear operator
          \[
            R(\lambda I - S_+)^{\perp} = N\p{\p{\lambda I - S_+}^*} = N\p{\overline{\lambda} I - S_-}
          \]
          Since $R(\lambda I - S_+)$ determines whether $\lambda$ is in the
          continuous spectrum or the residual spectrum, I will inspect
          $N\p{\overline{\lambda} I - S_-}$.
          Let $x \in N\p{\overline{\lambda} I - S_-}$, then
          \begin{align*}
            \p{\overline{\lambda} I - S_-}x &= 0 \\
            \overline{\lambda} x - S_-x &= 0 \\
            S_-x &= \overline{\lambda}x \\
            \p{S_-x}_n &= \overline{\lambda} x_n \\
            x_{n+1} &= \overline{\lambda} x_n \\
            \intertext{Inducting on this formula, we find that a explicit
              formula for $x_n$}
            x_n &= \overline{\lambda}^n x_0
          \end{align*}
          This sequence $x_n = \overline{\lambda}^n x_0$ for an arbitrary $x_0$
          is a potential element in $N\p{\overline{\lambda}I - S_-}$, yet it
          remains to be seen if $\set{x_n} \in \mcl^2$.
          In order to see if $\set{x_n}$ is in $\mcl^2$ consider
          $\sum{n = 0}{\infty}{\abs{x_n}^2}$.
          If this sum is convergent then $x = \set{x_n}$ is in $\mcl^2$ and if
          the sum is not convergent then $x$ is not in $\mcl^2$.
          \begin{align*}
            \sum{n = 0}{\infty}{\abs{x_n}^2} &= \sum{n = 0}{\infty}{\abs{\overline{\lambda}^n x_0}^2} \\
            &= \sum{n = 0}{\infty}{\abs{\overline{\lambda}}^{2n} \abs{x_0}^2} \\
            &= \abs{x_0}^2 \sum{n = 0}{\infty}{\abs{\overline{\lambda}}^{2n}} \\
          \end{align*}
          Since we already know that $\abs{\lambda} \le 1$, there are two
          possible cases, either $\abs{\lambda} = 1$ or $\abs{\lambda} < 1$.
          If $\abs{\lambda} < 1$, then
          \[
            \sum{n = 0}{\infty}{\abs{x_n}^2} = \abs{x_0}^2 \sum{n = 0}{\infty}{\abs{\overline{\lambda}}^{2n}}
            = \frac{\abs{x_0}^2}{1 - \abs{\overline{\lambda}}^2}
          \]
          as this is a geometric series.
          In this case, the sum converges for any $x_0$.
          Therefore for any $x_0$ the sequence $x_n = \overline{\lambda}^n x_0$
          is in $N\p{\overline{\lambda}I - S_-}$.
          This shows that $N\p{\overline{\lambda}I - S_-} \neq \set{0}$.
          If we examine the relationship with the range again we see that
          \begin{align*}
            R(\lambda I - S_+)^{\perp} &= N\p{\overline{\lambda}I - S_-} \\
            \p{R(\lambda I - S_+)^{\perp}}^{\perp} &= \p{N\p{\overline{\lambda}I - S_-}}^{\perp} \\
            \overline{R(\lambda I - S_+)} &\neq \p{\set{0}}^{\perp} \\
            \overline{R(\lambda I - S_+)} &\neq \mcl^2 \\
          \end{align*}
          Thus if $\abs{\lambda} < 1$, then $R(\lambda I - S_+)$ is not dense
          and $\lambda \in \sigma_r(S_+)$.
          This shows that
          $\set{\lambda \in \CC: \abs{\lambda} < 1} \subset \sigma_r(S_+)$.

          If one the other hand $\abs{\lambda} = 1$, then
          \[
            \sum{n = 0}{\infty}{\abs{x_n}^2} = \abs{x_0}^2 \sum{n = 0}{\infty}{\abs{\overline{\lambda}}^{2n}}
            = \abs{x_0}^2 \sum{n = 0}{\infty}{1}
          \]
          which only converges if $x_0 = 0$.
          Thus if $\abs{\lambda} = 1$, then
          $N\p{\overline{\lambda}I - S_-} = \set{0}$.
          Using the relationship with the range of $\lambda I - S_+$, we see
          that
          \begin{align*}
            R(\lambda I - S_+)^{\perp} &= N\p{\overline{\lambda}I - S_-} \\
            \p{R(\lambda I - S_+)^{\perp}}^{\perp} &= \p{N\p{\overline{\lambda}I - S_-}}^{\perp} \\
            \overline{R(\lambda I - S_+)} &= \p{\set{0}}^{\perp} \\
            \overline{R(\lambda I - S_+)} &= \mcl^2 \\
          \end{align*}
          This shows that the range of $\lambda I - S_+$ is dense in $\mcl^2$ when
          $\abs{\lambda} = 1$.

          Note that this shows that if $\lambda \in \sigma_r(S_+)$ then
          $\abs{\lambda} \not\ge 1$, so $\abs{\lambda} < 1$ and
          $\sigma_r(S_+) \subset \set{\lambda \in \CC: \abs{\lambda} < 1}$.
          Thus $\sigma_r(S_+) = \set{\lambda \in \CC: \abs{\lambda} < 1}$ in fact.

          Now if $\abs{\lambda} = 1$, then $R(\lambda I - S_+)$ is dense, so
          either $\lambda \in \sigma_c(S_+)$ or $\lambda \in \rho(S_+)$.
          Suppose $\lambda \in \rho(S_+)$, then because $\rho(S_+)$ must be an
          open set $\lambda$ must be an interior point of $\rho(S_+)$.
          Let $B$ be a ball of radius $\epsilon$ around $\lambda$.
          Since $\abs{\lambda} = 1$, there must be some $\mu \in B$ such that
          $1 - \epsilon < \abs{mu} < 1$.
          However we have previously seen that if $\abs{\mu} < 1$, then
          $\mu \in \sigma_r(S_+)$.
          This contradicts the fact that $\lambda$ is an interior point of
          $\rho(S_+)$, because any ball around $\lambda$ will contain points
          in the residual spectrum.
          Therefore $\lambda \not\in \rho(S_+)$ and $\lambda \in \sigma_c(S_+)$.
          Thus $\set{\lambda \in \CC: \abs{\lambda} = 1} \subset \sigma_c(S_+)$.
          But since $\sigma_r(S_+)$, $\sigma_c(S_+)$, and $\rho(S_+)$ are
          disjoint this shows that
          $\set{\lambda \in \CC: \abs{\lambda} = 1} = \sigma_c(S_+)$.
          This is because $\abs{\lambda} < 1$ implies $\lambda \in \sigma_r(S_+)$
          and $\abs{\lambda} > 1$ implies that $\lambda \in \rho(S_+)$.

          %Lastly since we have seen that
          %$\set{\lambda \in \CC: \abs{\lambda} > 1} \subset \rho(S_+)$,
          %$\set{\lambda \in \CC: \abs{\lambda} = 1} \subset \sigma_c(S_+)$, and
          %$\set{\lambda \in \CC: \abs{\lambda} < 1} = \sigma_r(S_+)$, and
          %we know that these sets partition the complex numbers, then equality
          %must hold.
          %That is these sets are not only subsets they are the resolvant set,
          %the continuous spectrum, and the residual spectrum.
          In conclusion this shows for part (b) that
          \[
            \set{\lambda \in \CC: \abs{\lambda} = 1} = \sigma_c(S_+)
          \]
          and for part (c) that
          \[
            \set{\lambda \in \CC: \abs{\lambda} < 1} = \sigma_c(S_+).
          \]
        \end{proof}
    \end{enumerate}

  \pagebreak
  \item[\#7] % Done
    If $\lambda \neq \pm 1, \pm i$ show that $\lambda$ is in the resolvant set
    of the Fourier Transform $\mcF$.
    (Suggestion: Assuming that a solution of $\mcF u - \lambda u = f$ exists,
    derive an explicit formula for it by justifying and using the identity
    \[
      \mcF^4 u = \lambda^4 u + \lambda^3 f + \lambda^2 \mcF f + \mcF^3 f
    \]
    together with the fact that $\mcF^4 = I$.)

    \begin{proof}
      Let $\lambda \in \CC$ such that $\lambda \neq \pm 1, \pm i$.
      We have previously shown that this implies that
      $\lambda \not\in \sigma_p(\mcF)$.
      Therefore $\lambda \in \rho(\mcF) \cup \sigma_c(\mcF) \cup \sigma_r(\mcF)$.
      This implies that for some $f \in L^2 \cap L^1$, there exists a solution
      $u \in L^2 \cap L^2$ such that $\mcF u - \lambda u = f$.
      This is equivalent to $\mcF u = f + \lambda u$.
      If we take the Fourier transform of each side several times the equality
      is perserved.
      \begin{align*}
        \mcF u &= \lambda u + f \\
        \mcF^2 u &= \lambda \mcF u + \mcF f \\
        \mcF^2 u &= \lambda \p{\lambda u + f} + \mcF f \\
        \mcF^2 u &= \lambda^2 u + \lambda f + \mcF f \\
        \mcF^3 u &= \lambda^2 \mcF u + \lambda \mcF f + \mcF^2 f \\
        \mcF^3 u &= \lambda^2 \p{\lambda u + f} + \lambda \mcF f + \mcF^2 f \\
        \mcF^3 u &= \lambda^3 u + \lambda^2 f + \lambda \mcF f + \mcF^2 f \\
        \mcF^4 u &= \lambda^3 \mcF u + \lambda^2 \mcF f + \lambda \mcF^2 f + \mcF^3 f \\
        \mcF^4 u &= \lambda^3 \p{\lambda u + f} + \lambda^2 \mcF f + \lambda \mcF^2 f + \mcF^3 f \\
        \mcF^4 u &= \lambda^4 u + \lambda^3 f + \lambda^2 \mcF f + \lambda \mcF^2 f + \mcF^3 f
        \intertext{Since $\mcF^4 = I$}
        u &= \lambda^4 u + \lambda^3 f + \lambda^2 \mcF f + \lambda \mcF^2 f + \mcF^3 f \\
        u - \lambda^4 u &= \lambda^3 f + \lambda^2 \mcF f + \lambda \mcF^2 f + \mcF^3 f \\
        u &= \frac{1}{1 - \lambda^4} \p{\lambda^3 f + \lambda^2 \mcF f + \lambda \mcF^2 f + \mcF^3 f}
      \end{align*}
      Since $f \in L^2 \cap \L^1$ and $\lambda \neq \pm 1, \pm i$ this formula
      for $u$ is well-defined.
      It remains to be seen that $u \in L^2 \cap L^2$.
      However $\mcF f \in L^2 \cap L^1$ so $u$ is the sum of functions in
      $L^2 \cap L^1$ so $u \in L^2 \cap L^1$.
      Therefore we have an explicit formula for $u$ given any $f$, this shows
      that $R(\lambda I - \mcF) = L^2 \cap L^1$, so $\lambda \in \rho(\mcF)$.
    \end{proof}

  \pagebreak
  \item[\#8]
    Let $\v{H} = L^2(0, 1)$, $T_1 u = T_2 u = T_3 u = u'$ on the domains
    \begin{align*}
      D(T_1) &= H^1(0, 1) \\
      D(T_2) &= \set{u \in H^1(0,1): u(0) = 0} \\
      D(T_3) &= \set{u \in H^1(0, 1): u(0) = u(1) = 0}
    \end{align*}
    \begin{enumerate}
      \item[(i)] % Done
        Show that $\sigma(T_1) = \sigma_p(T_1) = \CC$.

        \begin{proof}
          Let $\lambda \in \CC$ and consider the equation $T_1 u = \lambda u$.
          Let $u(x) = e^{\lambda x}$, and note that $u \in H^1(0, 1)$,
          because $u \in L^2(0, 1)$ and $u'(x) = \lambda e^{\lambda x} \in L^2(0, 1)$.
          This is true because
          \[
            \dintt{0}{1}{\abs{e^{\lambda x}}^2}{x} < \infty
          \]
          and
          \[
            \dintt{0}{1}{\abs{\lambda e^{\lambda x}}^2}{x} < \infty.
          \]
          Since $u' = \lambda e^{\lambda x}$ it is clear that $u' = \lambda u$.
          Therefore $u$ is a nonzero solution to $T_1 u = \lambda u$, and
          therefore $\lambda \in \sigma_p(T_1)$.
          This shows that $\sigma_p(T_1) = \CC$.
        \end{proof}

      \item[(ii)] % Done
        Show that $\sigma(T_2) = \varnothing$.

        \begin{proof}
          Let $\lambda \in \CC$, and suppose that $\lambda \in \sigma(T_2)$.
          As in part $(i)$, if there is a solution to $T_2 u = \lambda u$, then
          $u(x) = A e^{\lambda x}$ where $A \neq 0$ is some scalar.
          However any $u$ of this form is not in $D(T_2)$ because even though
          $u \in H^1(0, 1)$, we have $u(0) = A \neq 0$.
          Therefore $\lambda \not\in \sigma_p(T_2)$.
          This also shows that $\lambda I - T_2$ is a one-to-one function for
          any $\lambda$, because if $\lambda I - T_2$ was not one-to-one then
          $\lambda \in \sigma_p(T_2)$.
          Consider the equation $(T_2 - \lambda I) u = f$ for some
          $f \in L^2(0, 1)$.
          This is a differential equation that can be solved using an
          integrating factor.
          \begin{align*}
            (T_2 - \lambda I) u &= f \\
            T_2u - \lambda u &= f \\
            u' - \lambda u &= f
            \intertext{Let $\mu = e^{-\lambda x}$ and multiply both sides of
              the equation}
            e^{-\lambda x} u'- \lambda e^{-\lambda x} u &= e^{-\lambda x} f \\
            \d*{e^{-\lambda x} u}{x} &= e^{-\lambda x} f \\
            e^{-\lambda x} u &= \dintt{0}{x}{e^{-\lambda y} f}{y}\\
            u &= e^{\lambda x}\dintt{0}{x}{e^{-\lambda y} f}{y} \\
          \end{align*}
          Note that this is one possible solution to this equation.
          The important quality of this solution is that $u \in D(T_2)$.
          To see this note that
          \[
            u(0) = e^{\lambda x}\dintt{0}{0}{e^{-\lambda y} f}{y} = 0
          \]
          Also $u \in L^2(0, 1)$ as it is the product of functions in $L^2(0, 1)$.
          The derivative of $u$ also exists by the product rule,
          \[
            u'(x) = \lambda e^{\lambda x}\dintt{0}{x}{e^{-\lambda x} f}{x} + e^{\lambda x} \p{e^{-\lambda x} f(x) - f(0)}.
          \]
          This derivative is in $L^2(0, 1)$ as well.
          Therefore $u \in H^1(0, 1)$ and $u \in D(T_2)$.
          This shows that for any $f$ there is a solution $u \in D(T_2)$ or in
          other words $R(\lambda I - T_2) = L^2(0, 1)$ for all $\lambda$.
          Thus $\lambda \in \rho(T_2)$, and not in $\sigma_r(T_2)$ or $\sigma_c(T_2)$.
          This shows that the spectrum is empty, i.e. $\sigma(T_2) = \varnothing$.
        \end{proof}

      \item[(iii)]
        Show that $\sigma(T_3) = \sigma_r(T_3) = \CC$.

        \begin{proof}
          Let $\lambda \in \CC$,
          Again as in part (i) if $T_3 u = \lambda u$ is going to have a
          solution, then it must be in the form $u(x) = A e^{\lambda x}$.
          However if $u \in D(T_3)$, then $u(0) = u(1) = 0$ and this implies
          that $A = 0$ which makes $u(x) = 0$.
          Thus there is no nonzero solution to $T_3 u = \lambda u$.
          Thus for any $\lambda \in \CC$, the operator $T_3 - \lambda I$ is
          one-to-one.

          As is part (ii) we can say that the equation
          $T_3 u - \lambda u = f$ has a solution if and only if
          $\d*{e^{-\lambda x} u} = e^{-\lambda x} f$.
          Let $G(x)$ be any antiderivative of $e^{-\lambda x} f$, then
          $u = e^{\lambda x} G(x)$.
          So in order for $u$ to be in $D(T_3)$ this implies that
          $G(0) = G(1) = 0$.
          This can only happen if
          \[
            \dintt{0}{1}{f(x) e^{-\lambda x}}{x} = 0
          \]
          So there is a solution to $(\lambda I - T_3)u = f$ when
          $f$ has this property.
          The set of $f \in L^2(0, 1)$, which have this property are not a
          dense subset of $L^2(0, 1)$, so this implies that
          $\lambda \in \sigma_r(T_3)$.
          Since this was true for any $\lambda$, this means that
          \[
            \sigma_r(T_3) = \CC
          \]
        \end{proof}
    \end{enumerate}

  \pagebreak
  \item[\#10] % Done
    Let $Tu(x) = \dintt{0}{x}{K(x, y)u(y)}{y}$ be a Volterra intergral operator
    on $L^2(0, 1)$ with a bounded kernel, $\abs{K(x, y)} \le M$.
    Show that $\sigma(T) = \set{0}$.
    (There are several ways to show that $T$ has no nonzero eigenvalues.
    Here is one approach: Define the equivalent norm on $L^2(0, 1)$
    \[
      \norm[\theta]{u}^2 = \dintt{0}{1}{\abs{u(x)}^2e^{-2\theta x}}{x}
    \]
    and show that the supremum of $\frac{\norm[\theta]{Tu}}{\norm[\theta]{u}}$
    can be made arbitrarily small by choosing $\theta$ sufficiently large.

    \begin{proof}
      First I will show that $\norm[\theta]{u}$ is indeed a norm.
      Let $u = 0$ then
      \begin{align*}
        \norm[\theta]{0}^2 &= \dintt{0}{1}{\abs{0}^2e^{-2\theta x}}{x} \\
        &= \dintt{0}{1}{0}{x} = 0
      \end{align*}
      Now suppose $\norm[\theta]{u}^2 = 0$,
      \begin{align*}
        \norm[\theta]{u}^2 &= \dintt{0}{1}{\abs{u}^2e^{-2\theta x}}{x} \\
        0 &= \dintt{0}{1}{\abs{u}^2e^{-2\theta x}}{x}.
        \intertext{Since this is an integral of a nonnegative function it can
          only be zero if the integrand is zero, this implies that}
        u &= 0.
      \end{align*}
      This shows that $\norm[\theta]{u} = 0$ if and only if $u = 0$.

      Now suppose $u \in L^2(0, 1)$ and $\lambda$ is some scalar, then
      \begin{align*}
        \norm[\theta]{\lambda u} &= \sqrt{\dintt{0}{1}{\abs{\lambda u}^2e^{-2\theta x}}{x}} \\
        &= \sqrt{\abs{\lambda}^2 \dintt{0}{1}{\abs{u}^2e^{-2\theta x}}{x}} \\
        &= \abs{\lambda} \sqrt{\dintt{0}{1}{\abs{u}^2e^{-2\theta x}}{x}} \\
        &= \abs{\lambda} \norm[\theta]{u}
      \end{align*}
      This shows that $\norm[\theta]{\lambda u} = \abs{\lambda} \norm[\theta]{u}$.

      Let $u, v \in L^2(0, 1)$, then
      \begin{align*}
        \norm[\theta]{u + v} &= \sqrt{\dintt{0}{1}{\abs{u(x) + v(x)}^2e^{-2\theta x}}{x}} \\
        &= \sqrt{\dintt{0}{1}{\abs{u(x)e^{-\theta x} + v(x)e^{-\theta x}}^2}{x}} \\
        &= \norm[L^2(0, 1)]{u(x)e^{-\theta x} + v(x)e^{-\theta x}} \\
        &\le \norm[L^2(0, 1)]{u(x)e^{-\theta x}} + \norm[L^2]{v(x)e^{-\theta x}} \\
        &= \sqrt{\dintt{0}{1}{\abs{u(x)e^{-\theta x}}^2}{x}} + \sqrt{\dintt{0}{1}{\abs{u(x)e^{-\theta x}}^2}{x}} \\
        &= \sqrt{\dintt{0}{1}{\abs{u(x)}^2e^{-2\theta x}}{x}} + \sqrt{\dintt{0}{1}{\abs{u(x)}^2e^{-2\theta x}}{x}} \\
        &= \norm[\theta]{u(x)} + \norm[\theta]{u(x)}
      \end{align*}
      Thus this function also satisfies the triangle inequality and therefore
      it is a norm.

      Now that we have established that $\norm[\theta]{\cdot}$ is a norm, I will
      show that this norm is equivalent to the $L^2$ norm.
      First note that $e^{-2\theta} \le e^{-2\theta x} \le 1$ when
      $x \in \br{0, 1}$.
      Using this it is clear that
      \[
        e^{-2\theta} \dintt{0}{1}{\abs{u(x)}^2}{x} \le \dintt{0}{1}{\abs{u(x)}^2 e^{-2\theta x}}{x} \le \dintt{0}{1}{\abs{u(x)}^2}{x}
      \]
      or equivalently
      \[
        e^{-2\theta} \norm[L^2]{u} \le \norm[\theta]{u} \le \norm[L^2]{u}.
      \]
      for any $u \in L^2(0, 1)$.
      This shows that the norm $\norm[\theta]{\cdot}$ is equivalent to
      $\norm[L^2]{\cdot}$.

      Next I will show that $\sup{\frac{\norm[\theta]{Tu}}{\norm[\theta]{u}}}$
      arbitrarily small, by making $\theta$ large enough.
      Let $\epsilon > 0$ be fixed.
      \begin{align*}
        \norm[\theta]{Tu}^2 &= \dintt{0}{1}{\abs{Tu(x)}^2 e^{-2\theta x}}{x} \\
        &= \dintt{0}{1}{\abs{\dintt{0}{x}{K(x, y)u(y)}{y}}^2 e^{-2\theta x}}{x}
        \intertext{Using Holder's Inequality}
        &\le \dintt{0}{1}{e^{-2\theta x}}{x} \dintt{0}{1}{\dintt{0}{x}{\abs{K(x, y)u(y)}^2}{y}}{x}
        \intertext{Using Fubini's Theorem}
        &= \dintt{0}{1}{e^{-2\theta x}}{x} \dintt{0}{1}{\abs{u(y)}^2\dintt{y}{1}{\abs{K(x, y)}^2}{x}}{y}
        \intertext{Since $K(x, y)$ is bounded on $\br{0, 1} \times \br{0, 1}$, there exists $M$ such that}
        &\le \dintt{0}{1}{e^{-2\theta x}}{x} \dintt{0}{1}{M\abs{u(y)}^2}{y} \\
        &= \frac{M}{-2\theta e^{-2\theta}} \norm[\theta]{u}^2
      \end{align*}
      This implies that
      \[
        \frac{\norm[\theta]{Tu}}{\norm[\theta]{u}} \le \sqrt{\frac{M}{-2\theta e^{-2\theta}}}
      \]
      Since $\theta$ can be any number there exists $\theta$ such that
      \[
        \frac{\norm[\theta]{Tu}}{\norm[\theta]{u}} \le \epsilon
      \]

      Now that we have shown that
      \[
        \norm[\theta]{T} = \sup{\frac{\norm[\theta]{Tu}}{\norm[\theta]{u}}} < \epsilon
      \]
      for any $\epsilon > 0$, we can show that $\sigma(T) = \set{0}$.
      Theorem 12.1 states that $\abs{\lambda} \le \norm{T}$ for any
      $\lambda \in \sigma(T)$, because $T \in \mcB(L^2(0, 1))$.
      However since $\norm[\theta]{T}$ is equivalent to $\norm{T}$, this also
      means that $\abs{\lambda} \le \norm[\theta]{T}$ for any $\theta$.
      Now since $\norm[\theta]{T}$ can be arbitrarily small this implies that
      $\lambda = 0$ is the only possible element of $\sigma(T)$.
      Also by Theorem 12.3 we know that $\sigma(T)$ is nonempty because
      $T \in \mcB(L^2(0, 1))$.
      This shows that $\sigma(T) = \set{0}$.
    \end{proof}

  \pagebreak
  \item[\#11] % Done
    If $T$ is a symmetric operator, show that
    \[
      \sigma_p(T) \cup \sigma_c(T) \subset \RR
    \]
    (It is almost the same as showing that $\sigma(T) \subset \RR$ for a
    self-adjoint operator.)

    \begin{proof}
      Let $\lambda = \xi + i \eta$ with $\eta \neq 0$.
      Now consider for $u \in D(T)$,
      \begin{align*}
        \norm{\p{\lambda I - T}u}^2 &= \norm{\lambda u - Tu}^2 \\
        &= \abr{\lambda u - Tu, \lambda u - Tu} \\
        &= \abr{\xi u + i \eta u - Tu, \xi u + i \eta u - Tu} \\
        &= \abr{\xi u - Tu, \xi u - Tu} + \abr{i \eta u, \xi u - Tu} + \abr{\xi u - Tu, i \eta u} + \abr{i\eta u, i \eta u} \\
        &= \norm{\xi u - Tu}^2 + \abr{i \eta u, \xi u - Tu} + \abr{\xi u - Tu, i \eta u} + \norm{i \eta u}^2 \\
        &= \norm{\xi u - Tu}^2 + \abr{i \eta u, \xi u - Tu} + \abr{\xi u - Tu, i \eta u} + \abs{\eta}^2 \norm{u}^2
      \end{align*}
      Now note that
      \begin{align*}
        \abr{i \eta u, \xi u - Tu} + \abr{\xi u - Tu, i \eta u}
        &= i\eta\abr{u, \xi u - Tu} -i\eta \abr{\xi u - Tu,u} \\
        &= i\eta\p{\abr{u, \xi u - Tu} -\abr{\xi u - Tu,u}} \\
        &= i\eta\p{\abr{u, \xi u - Tu} -\p{\abr{\xi u,u} - \abr{Tu,u}}} \\
        &= i\eta\p{\abr{u, \xi u - Tu} -\p{\xi\abr{u,u} - \abr{Tu,u}}}
        \intertext{As $T$ is symmetric}
        &= i\eta\p{\abr{u, \xi u - Tu} -\p{\xi\abr{u,u} - \abr{u,Tu}}}
        \intertext{As $\xi$ is real}
        &= i\eta\p{\abr{u, \xi u - Tu} -\p{\abr{u,\xi u} - \abr{u,Tu}}} \\
        &= i\eta\p{\abr{u, \xi u - Tu} - \abr{u,\xi u - Tu}} \\
        &= i\eta\p{0} = 0
      \end{align*}
      Therefore
      \[
        \norm{\p{\lambda I - T}u}^2 = \norm{\xi u - Tu}^2 + \abs{\eta}^2 \norm{u}^2
      \]
      Now as $\norm{\xi u - Tu}^2 \ge 0$, this implies that
      \[
        \norm{\p{\lambda I - T}u}^2 \ge \abs{\eta}^2 \norm{u}^2
      \]
      or
      \[
        \norm{\p{\lambda I - T}u} \ge \abs{\eta} \norm{u}.
      \]
      Since $\abs{\eta} > 0$ this implies that $\lambda I - T$ is one-to-one
      because if $\norm{u} > 0$ then $\norm{\p{\lambda I - T}u} > 0$ which shows
      that $N(\lambda I - T) = \set{0}$.
      Since $\lambda I - T$ is one-to-one for any complex $\lambda$ this shows
      that $\lambda \not\in \sigma_p(T)$.
      Therefore $\sigma_p(T) \subset \RR$.

      Since $\lambda \not\in \sigma_p(T)$ this implies that
      $\lambda \in \sigma_c(T)$.
      Since $\lambda \in \sigma_c(T)$, then $\overline{R(\lambda I - T)} = H$.

      I now claim that $R(\lambda I - T)$ must be closed.
      To see this note that $(\lambda I - T)^{-1}$ is well-defined because
      $\lambda I - T$ is one-to-one.
      Also $(\lambda I - T)^{-1}$ is bounded.
      If $u \in D(\lambda I - T)$, then there exists
      $v \in R(\lambda I - T) = D\p{(\lambda I - T)^{-1}}$ such that
      $u = (\lambda I - T)^{-1} v$.
      Now
      \begin{align*}
        \norm{(\lambda I - T)u} &\ge \abs{\eta} \norm{u} \\
        \norm{(\lambda I - T)(\lambda I - T)^{-1}u} &\ge \abs{\eta} \norm{(\lambda I - T)^{-1}u} \\
        \norm{u} &\ge \abs{\eta} \norm{(\lambda I - T)^{-1}u} \\
        \frac{\norm{(\lambda I - T)^{-1}u}}{\norm{u}} &\le \frac{1}{\abs{\eta}} \\
        \norm{(\lambda I - T)^{-1}} &\le \frac{1}{\abs{\eta}} \\
      \end{align*}
      Thus this shows that $(\lambda I - T)^{-1}$ is bounded and continuous.
      Now to see that $R(\lambda I - T)$ is closed let $v_n \in R(\lambda I - T)$,
      such that $v_n \to v \in H$.
      Since $v_n \in R(\lambda I - T)$, there exists $u_n \in D(\lambda I - T)$
      such that $(\lambda I - T)u_n = v_n$.
      It can be shown that $\set{u_n}$ converges.
      To see this note that $\norm{v_n - v_m} \to 0$, this is equivalent to
      \begin{align*}
        \norm{(\lambda I - T)u_n - (\lambda I - T)u_m} &= \norm{(\lambda I - T)(u_n - u_m)} \\
        &\ge \abs{eta} \norm{u_n - u_m}
      \end{align*}
      Thus $\norm{u_n - u_m} \to 0$ as well.
      This shows that $\set{u_n}$ is Cauchy and since this is a Hilbert space
      the sequence is convergent.
      Therefore $u_n \to u$.
      Consider now
      \begin{align*}
        u &= \lim[n \to \infty]{u_n} \\
        &= \lim[n \to \infty]{(\lambda I - T)^{-1}(\lambda I - T)u_n}
        \intertext{Since $(\lambda I - T)$ is continuous}
        &= (\lambda I - T)^{-1} \lim[n \to \infty]{(\lambda I - T)u_n} \\
        &= (\lambda I - T)^{-1} \lim[n \to \infty]{v_n} \\
        &= (\lambda I - T)^{-1} v
      \end{align*}
      This shows that $u \in R((\lambda I - T)^{-1}) = D(\lambda I - T)$ and
      therefore $(\lambda I - T)u = v$.
      Therefore $v \in R(\lambda I - T)$, so $R(\lambda I - T)$ is closed.
      In conclusion since $R(\lambda I - T)$ is dense and closed, this implies
      that $R(\lambda I - T) = H$.
      This contradicts that $\lambda \in \sigma_c(T)$, so this shows that if
      $\lambda \in \CC$, then $\lambda \not\in \sigma_c(T)$.
      Therefore $\sigma_c(T) \subset \RR$.
    \end{proof}
\end{enumerate}
\end{document}
