\documentclass[11pt, oneside]{article}
\usepackage[letterpaper, margin=2cm]{geometry}
\usepackage{MATH520}

\begin{document}
\noindent \textbf{\Large{Caleb Logemann \\
MATH 520 Methods of Applied Math II \\
Homework 5
}}

\subsection*{Section 12.4}
\begin{enumerate}
  \item[\#6]
    Let $T$ denote the right shift operator on $\mcl^2$.
    \begin{enumerate}
      \item[(a)] % Done
        Show that $\sigma_p(T) = \varnothing$.

        \begin{proof}
          First I will let $S_+ = T$, so as to better represent the right shift
          operator.
          In order to show that $\sigma_p(S_+) = \varnothing$, we must show that
          $S_+$ has no eigenvalues.
          First I will show that $0$ is not an eigenvalue, if $0$ was an
          eigenvalue then $S_+ x = 0 x = 0$ would have a nonzero solution
          $x \in \mcl^2$.
          However the equation $S_+ x = 0$ guarantees that
          \[
            \p{S_+ x}_k = x_{k-1} = 0
          \]
          for $k \ge 1$, which implies that $x = 0$.
          This shows that the only solution to $S_+ x = 0$ is $x = 0$, so $0$
          is not an eigenvalue of $S_+$, e.g. $0 \not\in \sigma_p(S_+)$.

          Next I will show that no nonzero value can be an eigenvalue.
          Assume to the contrary that $\lambda \neq 0 \in \sigma_p(T)$, that is
          $\lambda$ is an eigenvalue of $T$.
          This implies that $S_+ x = \lambda x$ has a nonzero solution
          $x \in \mcl^2$.
          Both $S_+ x$ and $\lambda x$ are sequences and for the sequences to be
          equal each term in the sequences must be equal.
          Therefore I will compare the terms of these sequences.
          Note that by definition $\p{S_+ x}_k = x_{k-1}$ for $k \ge 1$ and
          $\p{S_+ x}_0 = 0$.
          Using this definition, in $S_+ x = \lambda x$ two conditions arise
          first for $k = 0$
          \[
            \lambda x_0 = \p{\lambda x}_0 = \p{S_+ x}_0 = 0
          \]
          and for $k \ge 1$
          \[
            \lambda x_k = \p{\lambda x}_k = \p{S_+ x}_k = x_{k-1}
          \]
          The first condition shows that $x_0 = 0$, and the second that
          $x_k = \frac{x_{k-1}}{\lambda}$.
          However these two statements together inductively show that $x_k = 0$
          for $k \ge 0$.
          Thus $x = 0$ is the only solution to the equation $S_+ x = \lambda x$,
          and $\lambda \not\in \sigma_p(S_+)$.
          This shows that no complex number can be an eigenvalue of $S_+$ and so
          $\sigma_p(S_+) = \varnothing$.
        \end{proof}

      \item[(b)]
        Show that $\sigma_c(T) = \set{\lambda: \abs{\lambda}= 1}$.

      \item[(c)]
        Show that $\sigma_r(T) = \set{\lambda: \abs{\lambda} < 1}$.

        \begin{proof}
          Again I will let $S_+ = T$ and I will prove (b) and (c) simultaneaously.
          From part (a) it is clear that for any $\lambda$, the operator
          $\lambda I - S_+$ is one-to-one.
          If $\lambda I - S_+$ was not one-to-one then
          $\lambda \in \sigma_p(S_+)$, however we have already shown that
          $\sigma_p(S_+)$ is empty.
          Thus for any $\lambda \in \CC$, $\lambda$ must be in the resolvant
          set, the continuous spectrum, or the residual spectrum.
          From Example 12.5 it is known that for $\abs{\lambda} > 1$, then
          $\lambda \in \rho(S_+)$.
          This is shown by noting that for bounded operators
          $\lambda \in \sigma(S_+)$ implies that
          $\abs{\lambda} \le \norm{S_+} = 1$.
          Let $\lambda \in \sigma(S_+)$, then $\abs{\lambda} \le 1$.
          Note that $I, S_+ \in \mcB(\mcl^2)$, so
          \[
            \p{\lambda I - S_+}^* = \overline{\lambda} I^* - S_+^* = \overline{\lambda} I - S_-.
          \]
          Also since $\lambda I - S_+$ is densely defined linear operator
          \[
            R(\lambda I - S_+)^{\perp} = N\p{\p{\lambda I - S_+}^*} = N\p{\overline{\lambda} I - S_-}
          \]
          Since $R(\lambda I - S_+)$ determines whether $\lambda$ is in the
          continuous spectrum or the residual spectrum, I will inspect
          $N\p{\overline{\lambda} I - S_-}$.
          Let $x \in N\p{\overline{\lambda} I - S_-}$, then
          \begin{align*}
            \p{\overline{\lambda} I - S_-}x &= 0 \\
            \overline{\lambda} x - S_-x &= 0 \\
            S_-x &= \overline{\lambda}x \\
            \p{S_-x}_n &= \overline{\lambda} x_n \\
            x_{n+1} &= \overline{\lambda} x_n \\
            \intertext{Inducting on this formula, we find that a explicit
              formula for $x_n$}
            x_n &= \overline{\lambda}^n x_0
          \end{align*}
          This sequence $x_n = \overline{\lambda}^n x_0$ for an arbitrary $x_0$
          is a potential element in $N\p{\overline{\lambda}I - S_-}$, yet it
          remains to be seen if $\set{x_n} \in \mcl^2$.
          In order to see if $\set{x_n}$ is in $\mcl^2$ consider
          $\sum{n = 0}{\infty}{\abs{x_n}^2}$.
          If this sum is convergent then $x = \set{x_n}$ is in $\mcl^2$ and if
          the sum is not convergent then $x$ is not in $\mcl^2$.
          \begin{align*}
            \sum{n = 0}{\infty}{\abs{x_n}^2} &= \sum{n = 0}{\infty}{\abs{\overline{\lambda}^n x_0}^2} \\
            &= \sum{n = 0}{\infty}{\abs{\overline{\lambda}}^{2n} \abs{x_0}^2} \\
            &= \abs{x_0}^2 \sum{n = 0}{\infty}{\abs{\overline{\lambda}}^{2n}} \\
          \end{align*}
          Since we already know that $\abs{\lambda} \le 1$, there are two
          possible cases, either $\abs{\lambda} = 1$ or $\abs{\lambda} < 1$.
          If $\abs{\lambda} = 1$, then
          \[
            \sum{n = 0}{\infty}{\abs{x_n}^2} = \abs{x_0}^2 \sum{n = 0}{\infty}{\abs{\overline{\lambda}}^{2n}}
            = \abs{x_0}^2 \sum{n = 0}{\infty}{1}
          \]
          which only converges if $x_0 = 0$.
          Thus if $\abs{\lambda} = 1$, then
          $N\p{\overline{\lambda}I - S_-} = \set{0}$.
          Using the relationship with the range of $\lamdba I - S_+$, we see
          that
          \begin{align*}
            R(\lambda I - S_+)^{\perp} &= N\p{\overline{\lambda}I - S_-} \\
            \p{R(\lambda I - S_+)^{\perp}}^{\perp} &= \p{N\p{\overline{\lambda}I - S_-}}^{\perp} \\
            \overline{R(\lambda I - S_+)} &= \p{\set{0}}^{\perp} \\
            \overline{R(\lambda I - S_+)} &= \mcl^2 \\
          \end{align*}
          This shows that the range of $\lambda I - S_+$ is dense in $\mcl^2$,
          so if $\abs{\lambda} = 1$, then $\lambda \in \sigma_c(S_+)$ or
          $\lambda \in \rho(S_+)$.

          If on the other hand $\abs{\lambda} < 1$, then
          \[
            \sum{n = 0}{\infty}{\abs{x_n}^2} = \abs{x_0}^2 \sum{n = 0}{\infty}{\abs{\overline{\lambda}}^{2n}}
            = \frac{\abs{x_0}^2}{1 - \abs{\overline{\lambda}}^2}
          \]
          as this is a geometric series.
          In this case, the sum converges for any $x_0$.
          Therefore for any $x_0$ the sequence $x_n = \overline{\lambda}^n x_0$
          is in $N\p{\overline{\lambda}I - S_-}$.
          This shows that $N\p{\overline{\lambda}I - S_-} \neq \set{0}$.
          If we examine the relationship with the range again we see that
          \begin{align*}
            R(\lambda I - S_+)^{\perp} &= N\p{\overline{\lambda}I - S_-} \\
            \p{R(\lambda I - S_+)^{\perp}}^{\perp} &= \p{N\p{\overline{\lambda}I - S_-}}^{\perp} \\
            \overline{R(\lambda I - S_+)} &\neq \p{\set{0}}^{\perp} \\
            \overline{R(\lambda I - S_+)} &\neq \mcl^2 \\
          \end{align*}
          Thus if $\abs{\lambda} < 1$, then $R(\lambda I - S_+)$ is not dense
          and $\lambda \in \sigma_r(S_+)$.
          This shows that
          $\set{\lambda \in \CC: \abs{\lambda} < 1} \subset \sigma_r(S_+)$.

          Lastly since we have seen that
          $\set{\lambda \in \CC: \abs{\lambda} > 1} \subset \rho(S_+)$,
          $\set{\lambda \in \CC: \abs{\lambda} = 1} \subset \sigma_c(S_+)$, and
          $\set{\lambda \in \CC: \abs{\lambda} < 1} \subset \sigma_r(S_+)$, and
          we know that these sets partition the complex numbers, then equality
          must hold.
          That is these sets are not only subsets they are the resolvant set,
          the continuous spectrum, and the residual spectrum.
          This shows for part (b) that
          \[
            \set{\lambda \in \CC: \abs{\lambda} = 1} = \sigma_c(S_+)
          \]
          and for part (c) that
          \[
            \set{\lambda \in \CC: \abs{\lambda} < 1} = \sigma_c(S_+).
          \]
        \end{proof}
    \end{enumerate}

  \pagebreak
  \item[\#7] % Done
    If $\lambda \neq \pm 1, \pm i$ show that $\lambda$ is in the resolvant set
    of the Fourier Transform $\mcF$.
    (Suggestion: Assuming that a solution of $\mcF u - \lambda u = f$ exists,
    derive an explicit formula for it by justifying and using the identity
    \[
      \mcF^4 u = \lambda^4 u + \lambda^3 f + \lambda^2 \mcF f + \mcF^3 f
    \]
    together with the fact that $\mcF^4 = I$.)

    \begin{proof}
      Let $\lambda \in \CC$ such that $\lambda \neq \om 1, \pm i$.
      We have previously shown that this implies that
      $\lambda \not\in \sigma_p(\mcF)$.
      Therefore $\lambda \in \rho(\mcF) \union \sigma_c(\mcF) \cup \sigma_r(\mcF)$.
      This implies that for some $f \in L^2 \cap L^1$, there exists a solution
      $u \in L^2 \cap L^2$ such that $\mcF u - \lambda u = f$.
      This is equivalent to $\mcF u = f + \lambda u$.
      If we take the Fourier transform of each side several times the equality
      is perserved.
      \begin{align*}
        \mcF u &= \lambda u + f \\
        \mcF^2 u &= \lambda \mcF u + \mcF f \\
        \mcF^2 u &= \lambda \p{\lambda u + f} + \mcF f \\
        \mcF^2 u &= \lambda^2 u + \lambda f + \mcF f \\
        \mcF^3 u &= \lambda^2 \mcF u + \lambda \mcF f + \mcF^2 f \\
        \mcF^3 u &= \lambda^2 \p{\lambda u + f} + \lambda \mcF f + \mcF^2 f \\
        \mcF^3 u &= \lambda^3 u + \lambda^2 f + \lambda \mcF f + \mcF^2 f \\
        \mcF^4 u &= \lambda^3 \mcF u + \lambda^2 \mcF f + \lambda \mcF^2 f + \mcF^3 f \\
        \mcF^4 u &= \lambda^3 \p{\lambda u + f} + \lambda^2 \mcF f + \lambda \mcF^2 f + \mcF^3 f \\
        \mcF^4 u &= \lambda^4 u + \lambda^3 f + \lambda^2 \mcF f + \lambda \mcF^2 f + \mcF^3 f
        \intertext{Since $\mcF^4 = I$}
        u &= \lambda^4 u + \lambda^3 f + \lambda^2 \mcF f + \lambda \mcF^2 f + \mcF^3 f \\
        u - \lambda^4 u &= \lambda^3 f + \lambda^2 \mcF f + \lambda \mcF^2 f + \mcF^3 f \\
        u &= \frac{1}{1 - \lambda^4} \p{\lambda^3 f + \lambda^2 \mcF f + \lambda \mcF^2 f + \mcF^3 f}
      \end{align*}
      Since $f \in L^2 \cap \L^1$ and $\lambda \neq \pm 1, \pm i$ this formula
      for $u$ is well-defined.
      It remains to be seen that $u \in L^2 \cap L^2$.
      However $\mcF f \in L^2 \cap L^1$ so $u$ is the sum of functions in
      $L^2 \cap L^1$ so $u \in L^2 \cap L^1$.
      Therefore we have an explicit formula for $u$ given any $f$, this shows
      that $R(\lambda I - \mcF) = L^2 \cap L^1$, so $\lambda \in \rho(\mcF)$.
    \end{proof}

  \pagebreak
  \item[\#8]
    Let $\v{H} = L^2(0, 1)$, $T_1 u = T_2 u = T_3 u = u'$ on the domains
    \begin{align*}
      D(T_1) &= H^1(0, 1) \\
      D(T_2) &= \set{u \in H^1(0,1): u(0) = 0} \\
      D(T_3) &= \set{u \in H^1(0, 1): u(0) = u(1) = 0}
    \end{align*}
    \begin{enumerate}
      \item[(i)] % Done
        Show that $\sigma(T_1) = \sigma_p(T_1) = \CC$.

        \begin{proof}
          Let $\lambda \in \CC$ and consider the equation $T_1 u = \lambda u$.
          Let $u(x) = e^{\lambda x}$, and note that $u \in H^1(0, 1)$,
          because $u \in L^2(0, 1)$ and $u'(x) = \lambda e^{\lambda x} \in L^2(0, 1)$.
          This is true because
          \[
            \dintt{0}{1}{\abs{e^{\lambda x}}^2}{x} < \infty
          \]
          and
          \[
            \dintt{0}{1}{\abs{\lambda e^{\lambda x}}^2}{x} < \infty.
          \]
          Since $u' = \lambda e^{\lambda x}$ it is clear that $u' = \lambda u$.
          Therefore $u$ is a nonzero solution to $T_1 u = \lambda u$, and
          therefore $\lambda \in \sigma_p(T_1)$.
          This shows that $\sigma_p(T_1) = \CC$.
        \end{proof}

      \item[(ii)]
        Show that $\sigma(T_2) = \varnothing$.

        \begin{proof}
          Let $\lambda \in \CC$ 
        \end{proof}

      \item[(iii)]
        Show that $\sigma(T_3) = \sigma_r(T_3) = \CC$.

        \begin{proof}
          
        \end{proof}
    \end{enumerate}

  \pagebreak
  \item[\#10]
    Let $Tu(x) = \dintt{0}{x}{K(x, y)u(y)}{u}$ be a Volterra intergral operator
    on $L^2(0, 1)$ with a bounded kernel, $\abs{K(x, y)} \le M$.
    Show that $\sigma(T) = \set{0}$.
    (There are several ways to show that $T$ has no nonzero eigenvalues.
    Here is one approach: Define the equivalent norm on $L^2(0, 1)$
    \[
      \norm[\theta]{u}^2 = \dintt{0}{1}{\abs{u(x)}^2e^{-2\theta x}}{x}
    \]
    and show that the supremum of $\frac{\norm[\theta]{Tu}}{\norm[\theta]{u}}$
    can be made arbitrarily small by choosing $\theta$ sufficiently large.

    \begin{proof}
      
    \end{proof}

  \pagebreak
  \item[\#11]
    If $T$ is a symmetric operator, show that
    \[
      \sigma_p(T) \cup \sigma_c(T) \subset \RR
    \]
    (IT is almost the same as showing that $\sigma(T) \subset \RR$ for a
    self-adjoint operator.)

    \begin{proof}
      Let $\lambda = \xi + i \eta$ with $\eta \neq 0$.
    \end{proof}
\end{enumerate}
\end{document}
