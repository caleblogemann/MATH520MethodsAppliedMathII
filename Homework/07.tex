\documentclass[11pt, oneside]{article}
\usepackage[letterpaper, margin=2cm]{geometry}
\usepackage{MATH520}

\begin{document}
\noindent \textbf{\Large{Caleb Logemann \\
MATH 520 Methods of Applied Math II \\
Homework 7
}}

\subsection*{Section 13.6}
\begin{enumerate}
  \item[\#6] % Done
    Let
    \[
      Tu(x) = \frac{1}{x} \dintt{0}{x}{u(y)}{y} \qquad u \in L^2(0, 1)
    \]
    Show that $(0, 2) \subset \sigma_p(T)$ and that $T$ is not compact.
    % Suggestion look for eigenfunctions of the form u(x) = x^{\alpha}

    \begin{proof}
      Let $\lambda \in (0, 2)$ and define $\alpha = \frac{1}{\lambda} - 1$.
      Since $\lambda \in (0, 2)$ this is well-defined and this implies that
      $\alpha \in (-1/2, \infty)$.
      Now let $u(x) = x^{\alpha}$, and first note that $u \in L^2(0, 1)$ for
      any $\alpha \in (-1/2, \infty)$.
      \begin{align*}
        \dintt{0}{1}{\abs{u(x)}^2}{x} &= \dintt{0}{1}{\abs{x^{\alpha}}^2}{x} \\
        &= \dintt{0}{1}{x^{2\alpha}}{x}
        \intertext{Since $\alpha > -1/2$, this implies that $2\alpha > -1$ and
          this integral can be evaluated using the power rule}
        &= \frac{1}{2\alpha + 1}\eval{x^{2\alpha + 1}}{x = 0}{1} \\
        &= \frac{1}{2\alpha + 1} \\
        &\le \infty
      \end{align*}
      This shows that $u \in L^2(0, 1)$.

      Now $Tu(x)$,
      \begin{align*}
        Tu(x) &= \frac{1}{x} \dintt{0}{x}{u(y)}{y} \\
        &= \frac{1}{x} \dintt{0}{x}{y^{\alpha}}{y} \\
        &= \frac{1}{x} \eval{\frac{1}{\alpha + 1} y^{\alpha + 1}}{y = 0}{x} \\
        &= \frac{1}{x} \frac{1}{\alpha + 1} x^{\alpha + 1} \\
        &= \frac{1}{\alpha + 1} x^{\alpha} \\
        &= \frac{1}{\alpha + 1} u(x) \\
        &= \lambda u(x) \\
      \end{align*}

      This shows that $\lambda$ and $u(x)$ are an eigenvalue and eigenfunction
      pair for $T$.
      Thus $\lambda \in \sigma_p(T)$ and $(0, 2) \subset \sigma_p(T)$
    \end{proof}

  \item[\#9]
    Let $T$ be the integral operator with kernel $K(x, y) = e^{-\abs{x - y}}$
    on $L^2(-1, 1)$.
    Find all the eigenvalues and eigenfunctions of $T$.
    Suggestion: $Tu = \lambda u$ is equivalent to an ODE problem.
    Don't forget about boundary conditions.
    the eigenvalues may need to be characterized in terms of the roots of a
    certain nonlinear functions.

    In order to find the eigenvalues and eigenfunctions of $T$ we must find
    solutions to $Tu = \lambda u$.
    \begin{align*}
      \lambda u(x) &= Tu(x) \\
      \lambda u(x) &= \dintt{-1}{1}{e^{-\abs{x - y}} u(y)}{y} \\
      \lambda u(x) &= \dintt{-1}{x}{e^{-\abs{x - y}} u(y)}{y} + \dintt{x}{1}{e^{-\abs{x - y}} u(y)}{y} \\
      \lambda u(x) &= \dintt{-1}{x}{e^{-\abs{x - y}} u(y)}{y} - \dintt{1}{x}{e^{-\abs{x - y}} u(y)}{y}
      \intertext{In the first integral $y < x$ and in the second integral $x < y$}
      \lambda u(x) &= \dintt{-1}{x}{e^{-x + y} u(y)}{y} - \dintt{1}{x}{e^{x - y} u(y)}{y} \\
      \lambda u(x) &= e^{-x}\dintt{-1}{x}{e^y u(y)}{y} - e^x\dintt{1}{x}{e^{-y} u(y)}{y}
      \intertext{Differentiating both sides}
      \lambda u'(x) &= -e^{-x}\dintt{-1}{x}{e^y u(y)}{y} + e^{-x}e^{x}u(x) - e^x\dintt{1}{x}{e^{-y} u(y)}{y} - e^xe^{-x} u(x) \\
      \lambda u'(x) &= -e^{-x}\dintt{-1}{x}{e^y u(y)}{y} + u(x) - e^x\dintt{1}{x}{e^{-y} u(y)}{y} - u(x) \\
      \lambda u'(x) &= -e^{-x}\dintt{-1}{x}{e^y u(y)}{y} - e^x\dintt{1}{x}{e^{-y} u(y)}{y} \\
    \end{align*}

  \item[\#10] % Done
    We say that $T \in \mcB(\v{H})$ is a positive operator if
    $\abr{Tx, x} \ge 0$ for all $x \in \v{H}$.
    If $T$ is a positive self-adjoint compact operator show that $T$ has a
    square root, more precisely there exists a compact self-adjoint operator $S$
    such that $S^2 = T$.
    (Suggestion: If $T = \sum{n = 1}{\infty}{\lambda_n P_n}$ try
    $S = \sum{n = 1}{\infty}{\sqrt{\lambda_n} P_n}$.
    In a similar manner, one can define other fractional powers of $T$.)

    \begin{proof}
      Let $T$ be a positive self-adjoint compact operator.
      Note that the eigenvalues of $T$ must all be nonnegative.
      To see this, let $u$ be an eigenvector of $T$ with eigenvalue $\lambda$,
      then
      \[
        0 < \abr{Tu, u} = \abr{\lambda u, u} = \lambda \norm{u}^2
      \]
      This implies that $\lambda \ge 0$, since $\norm{u}^2 \ge 0$.
      Also since $T$ is self-adjoint and compact, the set of eigenvectors make
      an orthonormal basis of $\v{H}$ and $T$ can be expressed as the following
      infinite sum
      \[
        T = \sum{n = 1}{\infty}{\lambda_n P_n}
      \]
      where $P_n$ is the projection operator onto the eigenvector $u_n$ and
      $\lambda_n$ is the corresponding eigenvalue.
      Since $\lambda_n \ge 0$ for all $n$, we can define the following operator.
      \[
        S = \sum{n = 1}{\infty}{\sqrt{\lambda} P_n}
      \]
      This operator is compact as can be seen by noting that
      \[
        S_N = \sum{n = 1}{N}{\sqrt{\lambda} P_n}
      \]
      is a sequence of compact operators that converge to $S$.
      That is $S_N \in \mcK(\v{H})$ as it has a finite dimensional range.
      Also $S_N \to S$ so $S \in \mcK(\v{H})$ as $\mcK(\v{H})$ is a closed
      subspace.

      Also $S$ is self adjoint because it is a sum of self-adjoint operators.
      Lastly $S^2 = T$ because $P_n P_n = P_n$ and $P_n P_m = 0$ when
      $n \neq m$.
    \end{proof}

  \item[\#14] % Done
    If $Q \in \mcB(\v{H})$ is a Fredholm operator of index zero, show that there
    exists a one-to-one operator $S \in \mcB(\v{H})$ and $T \in \mcK(\v{H})$
    such that $Q = S + T$.
    (Hint: Define $T = AP$ where $P$ is the orthogonal projecion onto $N(Q)$ and
    $A: N(Q) \to N(Q^*)$ is one-to-one and onto.)

    \begin{proof}
      Let $Q \in \mcB(\v{H})$ be a Fredholm operator of index zero.
      This implies that $\dim(N(Q)) = \dim(N(Q^*)) < \infty$ and that $R(Q)$
      is closed.
      Since $\dim(N(Q)) = \dim(N(Q^*))$ there exists a one-to-one and onto
      operator $A:N(Q) \to N(Q^*)$.
      Note that since $A$ is bounded because it is a linear operator from one
      finite dimensional space to another finite dimensional space, see
      exercise 10.9.6.
      Now let $P$ be the projection from $H$ to $N(Q)$.
      Since $P$ is a projection it is bounded and since $\dim(N(Q)) < \infty$,
      $P$ is compact.
      Define $T = AP$, which is compact as it is the composition of a bounded
      operator with a compact operator.

      Next let $S = Q - T$.
      Since $Q, T \in \mcB(\v{H})$ this implies that $S \in \mcB(\v{H})$.
      Lastly I will show that $S$ is one-to-one.
      Let $u \in \v{H}$ such that $Su = 0$.
      This implies that $Qu - Tu = 0$ or that $Qu = Tu$.
      Since $Qu = Tu$, this implies that $Tu \in R(Q)$ or since the
      range of $Q$ is closed, $T \in \overline{R(Q)}$.
      However it is known that $\overline{R(Q)} = N(Q^*)^{\perp}$, so therefore
      $Tu \in N(Q^*)^{\perp}$.
      One the other hand $Tu = APu$ and therefore $Tu \in R(A) = N(Q^*)$.
      Since $Tu \in N(Q^*)$ and $Tu \in N(Q^*)^{\perp}$ this implies that
      $Tu = 0$ and therefore $Qu = 0$ because $Qu = Tu$.
      Since $Qu = 0$ this shows that $u \in N(Q)$.
      Also with $Tu = 0$, this implies that $APu = 0$ and since $A$ is
      one-to-one, $Pu = 0$.
      Because $Pu = 0$ this implies that $u$ must be orthogonal to $N(Q)$ as
      $P$ is the projection onto $N(Q)$, therefore $u \in N(Q)^{\perp}$.
      Now that $u \in N(Q)$ and $u \in N(Q)^{\perp}$, this shows that $u = 0$.
      We have now shown that $N(S) = \set{0}$ and therefore $S$ is one-to-one.

      Thus there exists a one-to-one and bounded function $S$ and a compact
      function $T$, such that $Q = S + T$ for any Fredholm operator of index 0,
      $Q$.
    \end{proof}

\end{enumerate}
\end{document}
