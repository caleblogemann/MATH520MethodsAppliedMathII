\documentclass[11pt, oneside]{article}
\usepackage[letterpaper, margin=2cm]{geometry}
\usepackage{MATH520}

\begin{document}
\noindent \textbf{\Large{Caleb Logemann \\
MATH 520 Methods of Applied Math II \\
Homework 6
}}

\subsection*{Section 13.6}
\begin{enumerate}
  \item[\#1] % Done
    Show that if $S \in \mcB{\v{H}}$ and $T$ is compact, the $TS$ and $ST$ are
    compact.

    \begin{proof}
      Let $E \subset \v{H}$ be bounded.
      Since $S \in \mcB(\v{H})$, this implies that the set $S(E)$ is also
      bounded.
      Now since $T$ is compact the set $T(S(E))$ will be precompact.
      This shows that $TS(E) = T(S(E))$ is precompact and that therefore the
      operator $TS$ is compact.

      Also since $T$ is compact this implies that $T(E)$ is precompact or that
      $\overline{T(E)}$ is compact.
      Let $\set{u_n} \in T(E)$, then there exists a subsequence $\set{u_{n_j}}$
      such that $u_{n_j} \to u \in \overline{T(E)}$.
      Consider $\set{S(u_{n_j})} \in S(T(E))$ and note that since
      $S \in \mcB(\v{H})$ there exists $v \in \overline{S(T(E))}$ such that
      $S(u_{n_j}) \to v$.
      This shows that $S(T(E))$ is also precompact, and therefore $ST$ is a
      compact operator.
    \end{proof}

  \pagebreak
  \item[\#2] % Done
    If $T \in \mcB(\v{H})$ and $T^*T$ is compact, show that $T$ must be compact.
    Use this to show that if $T$ is compact then $T^*$ must also be compact.

    \begin{proof}
      Let $T \in \mcB(\v{H})$ such that $T^* T$ is compact and let
      $u_n \wto u$ in $\v{H}$
      Since $T^* T$ is compact, $T^* T u_n \to u$.
      To show that $T$ compact we wish to show $Tu_n \to Tu$.
      \begin{align*}
        \lim[n \to \infty]{\norm{Tu_n - Tu}^2} &= \lim[n \to \infty]{\abs{\abr{Tu_n - Tu, Tu_n - Tu}}} \\
        &= \lim[n \to \infty]{\abs{\abr{Tu_n - Tu, Tu_n - Tu}}} \\
        &= \lim[n \to \infty]{\abs{\abr{T^*Tu_n - T^*Tu, u_n - u}}} \\
        &= \lim[n \to \infty]{\abs{\abr{T^*Tu_n, u_n} - \abr{T^*Tu_n, u} - \abr{T^*Tu, u_n} - \abr{T^*Tu, u}}}
        \intertext{Since $u_n \wto u$ and $T^*Tu_n \to T^*Tu$}
        &= \abs{\abr{T^*Tu, u} - \abr{T^*Tu, u} - \abr{T^*Tu, u} - \abr{T^*Tu, u}}
        &= 0
      \end{align*}
      This shows that $Tu_n \to Tu$, so $T$ is a compact operator.

      Now let $T$ be a compact operator, then by problem 1 $TT^*$ is compact as
      $T^* \in \mcB(\v{H})$.
      Now since $T^{**} = T$, this implies $\p{T^*}^* T^*$ is compact as well.
      Therefore by first part of this problem $T^*$ is compact.
    \end{proof}

  \pagebreak
  \item[\#4]
    It $T \in \mcB(\v{H})$ is compact and $\v{H}$ is of infinite dimension,
    show that $0 \in \sigma(T)$.

    \begin{proof}
      Let $T \in \mcB(\v{H})$ is compact and let $\v{H}$ be of infinite
      dimension.
      
      

    \end{proof}

  \pagebreak
  \item[\#13]
    The concept of a Hilbert-Schmidt operator can be defined abstractly as
    follows.
    If $\v{H}$ is a separable Hilber space, we say that $T \in \mcB(\v{H})$ is
    Hilbert-Schmidt if
    \[
      \sum{n = 1}{\infty}{\norm{Tu_n}^2} < \infty
    \]
    for some orthonormal basis $\set{u_n}_{n = 1}^{\infty}$ of $\v{H}$.
    \begin{enumerate}
      \item[(a)] % Done
        Show that if $T$ is Hilbert-Schmidt then the sum must be finite for any
        orthonormal basis of $\v{H}$.

        \begin{proof}
          First note that given any element $x \in \v{H}$ and any orthonormal
          basis $\set{u_n}_{n = 1}^{\infty}$, $x$ can be represented as its
          projection onto the basis, that is
          \[
            x = \sum{n = 1}{\infty}{\abr{x, u_n} u_n}
          \]
          This relationship can be used to rewrite $\norm{x}^2$.
          \begin{align*}
            \norm{x}^2 &= \abr{x, x} \\
            &= \abr{x, \sum{n = 1}{\infty}{\abr{x, u_n} u_n}} \\
            &= \sum{n = 1}{\infty}{\abr{x, \abr{x, u_n} u_n}} \\
            &= \sum{n = 1}{\infty}{\overline{\abr{x, u_n}} \abr{x, u_n}} \\
            &= \sum{n = 1}{\infty}{\abs{\abr{x, u_n}}^2} \\
          \end{align*}
          Therefore $\norm{x}^2 = \sum{n = 1}{\infty}{\abs{\abr{x, u_n}}^2}$
          for any $x \in \v{H}$ and any orthonormal basis
          $\set{u_n}_{n = 1}^{\infty}$.

          Now I will show $\sum{n = 1}{\infty}{\norm{Tv_n}^2}$ is finite for any
          orthonormal basis $\set{v_n}_{n = 1}^{\infty}$ when $T$ is a
          Hilbert-Schmidt operator.
          Let $\set{v_n}_{n = 1}^{\infty}$ be an orthonormal basis in $\v{H}$,
          and let $T$ be a Hilbert-Schmidt operator, then there exists another
          orthonormal basis $\set{u_n}_{n = 1}^{\infty}$ such that
          \[
            \sum{n = 1}{\infty}{\norm{Tu_n}^2} < \infty
          \]
          Now since $\norm{Tv_n}^2 = \sum{m = 1}{\infty}{\abs{\abr{Tv_n, u_n}}^2}$,
          \begin{align*}
            \sum{n = 1}{\infty}{\norm{Tv_n}^2} &= \sum{n = 1}{\infty}{\sum{m = 1}{\infty}{\abs{\abr{Tv_n, u_m}}^2}}
            \intertext{Since $T \in \mcB{\v{H}}$, $T^*$ exists}
            &= \sum{n = 1}{\infty}{\sum{m = 1}{\infty}{\abs{\abr{v_n, T^* u_m}}^2}} \\
            &= \sum{n = 1}{\infty}{\sum{m = 1}{\infty}{\abs{\abr{T^*u_m, v_n}}^2}} \\
            &= \sum{m = 1}{\infty}{\sum{n = 1}{\infty}{\abs{\abr{T^*u_m, v_n}}^2}}
            \intertext{Since $\set{v_n}$ is an orthonormal basis, $\sum{n = 1}{\infty}{\abs{\abr{T^*u_m, v_n}}^2} = \norm{T^* u_m}^2$}
            &= \sum{m = 1}{\infty}{\norm{T^* u_m}^2} \\
            &= \sum{m = 1}{\infty}{\sum{n = 1}{\infty}{\abs{\abr{T^*u_m, u_n}}^2}} \\
            &= \sum{m = 1}{\infty}{\sum{n = 1}{\infty}{\abs{\abr{u_m, Tu_n}}^2}} \\
            &= \sum{m = 1}{\infty}{\sum{n = 1}{\infty}{\abs{\abr{Tu_n, u_m}}^2}} \\
            &= \sum{n = 1}{\infty}{\sum{m = 1}{\infty}{\abs{\abr{Tu_n, u_m}}^2}} \\
            &= \sum{n = 1}{\infty}{\norm{Tu_n}^2} \\
            &< \infty
          \end{align*}
          This shows that $\sum{n = 1}{\infty}{\norm{Tv_n}^2} < \infty$ for any
          orthonormal basis.
        \end{proof}

      \item[(b)]
        Show that a Hilbert-Schmidt operator is compact.

        \begin{proof}
          Let $T$ be a Hilbert-Schmidt operator.
          Since $\mcK(\v{H})$ is a closed subspace of $\mcB(\v{H})$, if there
          exists some sequence of operators $T_N \in \mcK(\v{H})$ such that
          $T_N \to T$, then $T \in \mcK(\v{H})$ because $\mcK(\v{H})$ is closed.
          To this end, I will let $\set{u_n}_{n = 1}^{\infty}$ be an orthonormal
          basis of $\v{H}$ and I will define
          \[
            T_N x = \sum{n = 1}{N}{\abr{x, u_n} Tu_n}
          \]
          First I will show that $T_N \in \mcK(\v{H})$ for any $N$.
          Note that $T_N \in \mcB(\v{H})$,
          \begin{align*}
            \norm{T_N x} &= \norm{\sum{n = 1}{N}{\abr{x, u_n} Tu_n}} \\
            &\le \sum{n = 1}{N}{\abs{\abr{x, u_n}} \norm{Tu_n}}
            \intertext{By Cauchy-Schwarze}
            &\le \sum{n = 1}{N}{\norm{x} \norm{u_n} \norm{Tu_n}} \\
            &= \sum{n = 1}{N}{\norm{x} \norm{Tu_n}}
            \intertext{Since $T \in \mcB(\v{H})$}
            &\le \sum{n = 1}{N}{\norm{x} \norm{T}\norm{u_n}} \\
            &= \sum{n = 1}{N}{\norm{x} \norm{T}} \\
            &= N\norm{T}\norm{x} \\
          \end{align*}
          This shows that $\norm{T_N} \le N\norm{T}$ and therefore that
          $T_N \in \mcB(\v{H})$.
          Next note that $T_N x \in \spanspace{\set{Tu_n: 1 \le n \le N}}$.
          This shows that $R(T_N) \subset \spanspace{\set{Tu_n: 1 \le n \le N}}$ and
          that $\dim(R(T_N)) < N$.
          Since $T_N$ is both bounded and of finite rank, this implies that
          $T_N \in \mcK(\v{H})$.

          Secondly I will show that $T_N \to T$, and I will use the fact that
          \[
            Tx = T\p{\sum{n = 1}{\infty}{\abr{x, u_n}u_n}} = \sum{n = 1}{\infty}{\abr{x, u_n}Tu_n}
          \]
          \begin{align*}
            \norm{T_N - T} &= \sup[\norm{x} = 1]{\norm{T_N x - Tx}} \\
            &= \sup[\norm{x} = 1]{\norm{\sum{n = 1}{N}{\abr{x, u_n}Tu_n} - \sum{n = 1}{\infty}{\abr{x, u_n}Tu_n}}} \\
            &= \sup[\norm{x} = 1]{\norm{\sum{n = N+1}{\infty}{\abr{x, u_n}Tu_n}}} \\
            &\le \sup[\norm{x} = 1]{\sum{n = N+1}{\infty}{\abs{\abr{x, u_n}}\norm{Tu_n}}} \\
            &\le \sup[\norm{x} = 1]{\sum{n = N+1}{\infty}{\norm{x} \norm{u_n}\norm{Tu_n}}} \\
            &= \sum{n = N+1}{\infty}{\norm{Tu_n}} \\
          \end{align*}
        \end{proof}
    \end{enumerate}
\end{enumerate}
\end{document}
