\documentclass[11pt, oneside]{article}
\usepackage[letterpaper, margin=2cm]{geometry}
\usepackage{MATH520}

\begin{document}
\noindent \textbf{\Large{Caleb Logemann \\
MATH 520 Methods of Applied Math II \\
Homework 9
}}

\subsection*{Section 14.5}
\begin{enumerate}
  \item[\#5] % Done
    Let $Lu = a_2(x) u'' + a_1(x)u' + a_0(x) u$ with $a_2' = a$, so that $L$ is
    formally self adjoint.
    If $B_1 u = C_1 u(a) + C_2u'(a)$, $B_2 u = C_3u(b) + C_4u'(b)$, show that
    $\set{B_1^*, B_2^*} = \set{B_1, B_2}$.

    \begin{proof}
      Consider the operators $B_1^* \psi = C_1 \psi(a) + C_2 \psi'(a) = B_1 \psi$
      and $B_2^* \psi = C_3 \psi(b) + C_4 \psi'(b) = B_2 \psi$.
      In order to show that $\set{B_1^*, B_2^*}$ is adjoint to $\set{B_1, B_2}$
      it must be shown that
      \[
        \eval{J(\phi, \psi)}{a}{b} = 0
      \]
      whenever $B_1 \phi = B_2 \phi = B_1^* \psi = B_2^* \psi = 0$.

      Therefore let $\phi$ and $\psi$ be chosen such that
      \[
        B_1 \phi = B_2 \phi = B_1^* \psi = B_2^* \psi = 0.
      \]

      First consider $B_1 \phi = B_1^* \psi = 0$.
      These equations can be rewritten as
      \begin{align*}
        C_1 \phi(a) &= - C_2 \phi'(a) \\
        C_1 \psi(a) &= - C_2 \psi'(a).
      \end{align*}
      Multiplying these equations gives
      \begin{align*}
        -C_1 C_2 \phi(a) \psi'(a) &= -C_1 C_2 \psi(a) \phi'(a)
        \intertext{or}
        \phi(a) \psi'(a) - \phi'(a) \psi(a) &= 0
      \end{align*}
      Note that this statement is still true if one of $C_1$ or $C_2$ is
      zero.
      If one of them is zero, then either $\phi(a) = \psi(a) = 0$ or
      $\phi'(a) = \psi'(a) = 0$, which still gives
      \[
        \phi(a) \psi'(a) - \phi'(a) \psi(a) = 0 - 0 = 0
      \]

      Similary with $B_2 \phi = B_2^* \psi = 0$, it is possible to rewrite
      these equations as
      \begin{align*}
        C_3 \phi(b) &= - C_4 \phi'(b) \\
        C_3 \psi(b) &= - C_4 \psi'(b).
      \end{align*}
      Multiplying these equations gives
      \begin{align*}
        -C_3 C_4 \phi(b) \psi'(b) &= -C_3 C_4 \psi(b) \phi'(b)
        \intertext{or}
        \phi(b) \psi'(b) - \phi'(b) \psi(b) &= 0
      \end{align*}
      Again if $C_3$ or $C_4$ is zero, then $\phi(b) = \psi(b) = 0$ or
      $\phi'(b) = \psi'(a) = 0$ which implies
      \[
        \phi(b) \psi'(b) - \phi'(b) \psi(b) = 0
      \]

      Now consider the boundary functional $J(\phi, \psi)$.
      The boundary function $J(\phi, \psi)$ can be fully expressed as
      \[
        J(\phi, \psi) = a_2\p{\phi' \psi - \phi \psi'} + (a_1 - a_2') \phi \psi.
      \]
      Since the differential operator is formally self-adjoint, $a_1 = a_2'$, so
      this is equivalent to
      \[
        J(\phi, \psi) = a_2\p{\phi' \psi - \phi \psi'}
      \]

      Now the condition $\eval{J(\phi, \psi)}{a}{b} = 0$ is equivalent to
      \[
        a_2\p{\phi'(b) \psi(b) - \phi(b) \psi'(b)} - a_2\p{\phi'(a) \psi(a) - \phi(a) \psi'(a)} = 0.
      \]
      Since we have already shown that $\phi'(b) \psi(b) - \phi(b) \psi'(b) = 0$
      and $\phi'(a) \psi(a) - \phi(a) \psi'(a) = 0$ when the boundary operators
      are met.
      This condition is cleary met, when $B_1 \phi = B_2 \phi = B_1^* \psi = B_2^* \psi = 0$.

      Therefore $\set{B_1^*, B_2^*}$ is adjoint to $\set{B_1, B_2}$ and since
      $B_1 = B_1^*$ and $B_2 = B_2^*$ this implies that
      $\set{B_1^*, B_2^*} = \set{B_1, B_2}$.
      In other words $\set{B_1, B_2}$ is adjoint to itself.
      The purpose of this exercise is to show that a formally self-adjoint
      differential operator with boundary conditions of the form found in $B_1$
      and $B_2$ forms a self-adjoint linear operator.
    \end{proof}

  \pagebreak
  \item[\#8]
    When we rewrite $a_2(x) u'' + a_1(x) u' + a_0(x) u = \lambda u$ as
    \[
      -(p(x)u')' + q(x)u = \lambda \rho(x) u
    \]
    the latter is often referred to as the \textit{Liouville normal form}.
    Consider the eigenvalue problem
    \[
      x^2 u'' + xu' + u = \lambda u \qquad 1 < x < 2
    \]
    \[
      u(1) = u(2) = 0
    \]
    \begin{enumerate}
      \item[(a)] % Done
        Find the Liouville normal form.

        In order to find the Liouville normal form, the function $a_2(x)$ must
        be strictly less than zero, so I will first rewrite this eigenvalue
        problem as
        \[
          -x^2 u'' - xu' - u = -\lambda u \qquad 1 < x < 2
        \]
        \[
          u(1) = u(2) = 0
        \]
        The functions $p(x)$, $\rho(x)$, and $q(x)$ can be found as follows.
        \begin{align*}
          p(x) &= \exp\p{\dintt{a}{x}{\frac{a_1(s)}{a_2(s)}}{s}} \\
          &= \exp\p{\dintt{a}{x}{\frac{-s}{-s^2}}{s}} \\
          &= \exp\p{\dintt{a}{x}{\frac{1}{s}}{s}} \\
          &= \exp\p{\eval{\ln{s}}{s = a}{x}} \\
          &= e^{\ln{x} - \ln{a}} \\
          &= e^{\ln{\frac{x}{a}}} \\
          &= \frac{x}{a} \\
          \rho(x) &= - \frac{p(x)}{a_2(x)} \\
          &= - \frac{x/a}{-x^2} \\
          &= \frac{1}{ax} \\
          q(x) &= a_0(x) \rho(x) \\
          &= (-1)\frac{1}{ax}\\
          &= -\frac{1}{ax}\\
        \end{align*}
        Therefore the Liouville normal form of this eigenvalue problem is
        \[
          -\p{\frac{x}{a} \phi'}' - \frac{1}{ax} \phi = -\lambda \frac{1}{ax} \phi
        \]
        or
        \[
          \p{\frac{x}{a} \phi'}' + \frac{1}{ax} \phi = \lambda \frac{1}{ax} \phi
        \]

      \item[(b)] % Done
        What is the orthogonality relationship satisfied by the eigenfunctions?

        The eigenfunctions of this linear operator satisfy an orthogonality
        relationship with respect to the weight $\rho$.
        In mathematical terms,
        \[
          \dintt{a}{b}{\phi_n(x) \phi_m(x) \rho(x)}{x} =
          \begin{cases}
            0 & n \neq m \\
            1 & n = m
          \end{cases}
        \]
        or
        \[
          \dintt{a}{b}{\frac{\phi_n(x) \phi_m(x)}{ax}}{x} =
          \begin{cases}
            0 & n \neq m \\
            1 & n = m
          \end{cases}
        \]

      \item[(c)]
        Find the eigenvalues and eigenfunctions.
        (You may find the original form of the equation easier to work with than
        the Liouville normal form when computing the eigenvalues and
        eigenfunctions.)

    \end{enumerate}

  \pagebreak
  \item[\#10]
    Consider the Sturm-Liouville problem
    \begin{align*}
      u'' + \lambda u = 0 \qquad 0 < x < 1 \\
      u(0) - u'(0) = u(1) = 0
    \end{align*}
    \begin{enumerate}
      \item[(a)] % Done
        Multiply the equation by $u$ and integrate by parts to show that any
        eigenvalue is positive.

        First I will note a few useful facts, first since $u(0) - u'(0) = 0$,
        this implies that $u(0) = u'(0)$.
        Also if $u$ is nontrivial this guarantees that
        $\dintt{0}{1}{u^2(x)}{x} > 0$.
        Finally if $u$ is a nontrivial solution then $u'(x) \neq 0$ as
        $u(1) = 0$ makes any constant function is zero.
        This shows that $\dintt{0}{1}{\p{u'(x)}^2}{x} > 0$ as well.

        Multiplying by $u$ gives the following equation
        \[
          uu'' + \lambda u^2 = 0.
        \]
        Integrating both sides over $\br{0, 1}$ gives
        \[
          \dintt{0}{1}{u(x)u''(x)}{x} + \lambda \dintt{0}{1}{u^2(x)}{x} = \dintt{0}{1}{0}{x}
        \]
        This can be simplified using integration by parts
        \begin{align*}
          \dintt{0}{1}{u(x)u''(x)}{x} + \lambda \dintt{0}{1}{u^2(x)}{x} &= 0 \\
          \eval{u(x)u'(x)}{x = 0}{1} - \dintt{0}{1}{\p{u'(x)}^2}{x} + \lambda \dintt{0}{1}{u^2(x)}{x} &= 0 \\
          u(1)u'(1) - u(0)u'(0) - \dintt{0}{1}{\p{u'(x)}^2}{x} + \lambda \dintt{0}{1}{u^2(x)}{x} &= 0
          \intertext{Since $u(1) = 0$ and $u(0) = u'(0)$}
           -u^2(0) - \dintt{0}{1}{\p{u'(x)}^2}{x} + \lambda \dintt{0}{1}{u^2(x)}{x} &= 0.
        \end{align*}
        Since $\dintt{0}{1}{u^2(x)}{x} > 0$
        \[
          \lambda = \frac{u^2(0) + \dintt{0}{1}{\p{u'(x)}^2}{x}}{\dintt{0}{1}{u^2(x)}{x}} > 0.
        \]

      \item[(b)]
        Show that the eigenvalues are the positive solutions of
        $\tan{\sqrt{\lambda}} = - \sqrt{\lambda}$.

      \item[(c)] % Done
        Show graphically that such roots exist, and form an infinite sequence
        $\lambda_k$ such that $(k - 1/2)\pi < \sqrt{\lambda_k} < k \pi$ and
        \[
          \lim[k \to \infty]{\sqrt{\lambda_k} - (k - 1/2)\pi} = 0
        \]

        First this graph shows that solutions to the equation
        $\tan{\sqrt{\lambda}} = -\sqrt{\lambda}$ exist.
        \begin{center}
          \includegraphics[scale=.5]{Figures/09_1}
        \end{center}

        Next consider the sequence $\lambda_k = \p{(k - 1/2)\pi + \frac{1}{k}}^2$
        meets these criterion.
        Note that $\sqrt{\lambda_k} = (k - 1/2)\pi + \frac{1}{k}$.
        Clearly $\sqrt{\lambda_k} > (k - 1/2)\pi$ and $\sqrt{\lambda_k} < k\pi$
        as $\frac{1}{k} < \frac{\pi}{2}$.

        Also
        \begin{align*}
          \lim[k \to \infty]{\sqrt{\lambda_k} - (k - 1/2)\pi} &= \lim[k \to \infty]{(k - 1/2)\pi + \frac{1}{k} - (k - 1/2)\pi} \\
          &= \lim[k \to \infty]{\frac{1}{k}} = 0
        \end{align*}
    \end{enumerate}

  \pagebreak
  \item[\#14] % Done
    If $\set{\psi_n}_{n = 1}^{\infty}$ are Dirichlet eigenfunctions of the
    Laplacian making up an orthonormal basis of $L^2(\Omega)$, let
    $\zeta_n = \psi_n/\sqrt{\lambda_n}$
    ($\lambda_n$ the corresponding eigenvalue).
    \begin{enumerate}
      \item[(a)] % Done
        Show that $\set{\zeta_n}_{n = 1}^{\infty}$ is an orthonormal basis of
        $H^1_0(\Omega)$.

        \begin{proof}
          First I will show that $\zeta_n$ is an orthonormal set.
          In order to do this I will use that fact that
          \[
            \dintt{\Omega}{}{\nabla u \nabla v}{x} = \lambda \dintt{\Omega}{}{u v}{x}
          \]
          for any $v \in H_0^1(\Omega)$ where $\lambda$ and $u$ are a Dirichlet
          eigenvalue and eigenvector pair for the Laplacian on $\Omega$.

          Consider $\abr[H_0^1(\Omega)]{\zeta_n, \zeta_m}$
          \begin{align*}
            \abr[H_0^1(\Omega)]{\zeta_n, \zeta_m} &= \dintt{\Omega}{}{\nabla \zeta_n \cdot \nabla \zeta_m}{x} \\
            &= \frac{1}{\sqrt{\lambda_n \lambda_m}}\dintt{\Omega}{}{\nabla \psi_n \cdot \nabla \psi_m}{x}
            \intertext{Since $\psi_m \in H^1_0(\Omega)$ and $\psi_n$ an
              eigenfunction}
            &= \frac{\lambda_n}{\sqrt{\lambda_n \lambda_m}}\dintt{\Omega}{}{\psi_n \psi_m}{x} \\
            &= \frac{\lambda_n}{\sqrt{\lambda_n \lambda_m}} \abr[L^2(\Omega)]{\psi_n, \psi_m}.
            \intertext{Since $\set{\psi}$ already forms an orthonormal basis of
              $L^2$, so}
            &= \frac{\lambda_n}{\sqrt{\lambda_n \lambda_m}} \delta_{nm}
          \end{align*}
          where $\delta_{nm}$ is the Kronecker delta.
          This shows that if $n = m$, then
          $\abr[H_0^1(\Omega)]{\zeta_n, \zeta_m} = 1$ and if $n \neq m$, then
          $\abr[H_0^1(\Omega)]{\zeta_n, \zeta_m} = 0$.
          This shows that $\set{\zeta_n}$ is an orthonormal set in
          $H^1_0(\Omega)$.

          Now I will show that $\set{\zeta_n}$ is a basis of $H^1_0(\Omega)$.
          In order to show that this is a basis, I will show that this set is
          complete in $H^1_0(\Omega)$ or equivalently that the zero function is
          the only function orthogonal to the entire set.
          Therefore let $u \in H^1_0(\Omega)$, such that
          \[
            \abr[H^1_0(\Omega)]{\zeta_n, u} = 0
          \]
          for all $n \in \NN$.
          \begin{align*}
            0 &= \abr[H^1_0(\Omega)]{\zeta_n, u} \\
            &= \dintt{\Omega}{}{\nabla \zeta_n \cdot \nabla u}{x} \\
            &= \frac{1}{\sqrt{\lambda_n}}\dintt{\Omega}{}{\nabla \psi_n \cdot \nabla u}{x}
            \intertext{Since $u \in H^1_0(\Omega)$ and $\psi_n$ is a Dirichlet
              eigenfunction.}
            &= \frac{\lambda_n}{\sqrt{\lambda_n}}\dintt{\Omega}{}{\psi_n u}{x} \\
            &= \frac{\lambda_n}{\sqrt{\lambda_n}} \abr[L^2(\Omega)]{\psi_n, u}
          \end{align*}
          This shows that $\abr[L^2(\Omega)]{\psi_n, u} = 0$ for all $n$.
          Since $\set{\psi_n}$ is already a basis of $L^2$ this means that
          $u = 0$.
          Therefore $\set{\zeta_n}$ is complete in $H^1_0(\Omega)$ and thus
          is a basis of $H^1_0(\Omega)$.
        \end{proof}

      \item[(b)] % Done
        Show that $f \in H_0^1(\Omega)$ if and only if
        $\sum{n = 1}{\infty}{\lambda_n \abs{\abr{f, \psi_n}}^2} < \infty$

        \begin{proof}
          First I will do some manipulation on the given sum.
          \begin{align*}
            \sum{n = 1}{\infty}{\lambda_n \abs{\abr{f, \psi_n}}^2} &= \sum{n = 1}{\infty}{\lambda_n \abs{\dintt{\Omega}{}{f \psi_n}{x}}^2} \\
            &= \sum{n = 1}{\infty}{\frac{1}{\lambda_n} \abs{\dintt{\Omega}{}{f \lambda_n \psi_n}{x}}^2}
            \intertext{Since $\psi_n$ and $\lambda_n$ are an eigenfunction and
              eigenvalue pair for the Laplacian}
            &= \sum{n = 1}{\infty}{\frac{1}{\lambda_n} \abs{\dintt{\Omega}{}{f \Delta \psi_n}{x}}^2} \\
            &= \sum{n = 1}{\infty}{\frac{1}{\lambda_n} \abs{\dintt{\Omega}{}{\nabla f \nabla \psi_n}{x}}^2} \\
            &= \sum{n = 1}{\infty}{\frac{1}{\lambda_n} \abs{\abr[H^1_0(\Omega)]{f, \psi_n}}^2} \\
            &= \sum{n = 1}{\infty}{\abs{\abr[H^1_0(\Omega)]{f, \psi_n/\sqrt{\lambda_n}}}^2} \\
            &= \sum{n = 1}{\infty}{\abs{\abr[H^1_0(\Omega)]{f, \zeta_n}}^2} \\
          \end{align*}

          Now if $f \in H_0^1(\Omega)$ then 
          \[
            \sum{n = 1}{\infty}{\abs{\abr[H^1_0(\Omega)]{f, \zeta_n}}^2} < \infty
          \]
          and equivalently
          \[
            \sum{n = 1}{\infty}{\lambda_n \abs{\abr{f, \psi_n}}^2} < \infty.
          \]
          If on the other hand
          \[
            \sum{n = 1}{\infty}{\lambda_n \abs{\abr{f, \psi_n}}^2} < \infty.
          \]
          then
          \[
            \sum{n = 1}{\infty}{\abs{\abr[H^1_0(\Omega)]{f, \zeta_n}}^2} < \infty
          \]
          which implies that $f \in H^1_0(\Omega)$.
        \end{proof}
    \end{enumerate}

  \pagebreak
  \item[\#15]
    If $\Omega < \RR^n$ is a bounded open set with smooth enough boundary, find
    a solution of the wave equation problem
    \begin{align*}
      u_{tt} - \Delta u = 0 \qquad x \in \Omega \quad t > 0 \\
      u(x, t) = 0 \qquad x \in \partial\Omega \quad t > 0 \\
      u(x, 0) = f(x) \quad u_t(x, 0) = g(x) \qquad x \in \Omega
    \end{align*}
    in the form
    \[
      u(x, t) = \sum{n = 1}{\infty}{c_n(t) \psi_n(x)}
    \]
    where $\set{\psi_n}_{n = 1}^{\infty}$ are the Dirichlet eigenfunctions of
    $-\Delta$ in $\Omega$.

  \pagebreak
  \item[\#16]
    Derive formally that
    \[
      G(x, y) = \sum{n=1}{\infty}{\frac{\psi_n(x) \psi_n(y)}{\lambda_n}}
    \]
    where $\lambda_n, \psi_n$ are the Dirichlet eigenvalues and normalized
    eigenfunctions for the domain $\Omega$, and $G(x, y)$ is the corresponding
    Green's function in (14.4.96).
    (Suggestion: if $-\Delta u = f$, expand both $u$ and $f$ in the $\psi_n$
    basis.)

\end{enumerate}
\end{document}
