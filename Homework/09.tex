\documentclass[11pt, oneside]{article}
\usepackage[letterpaper, margin=2cm]{geometry}
\usepackage{MATH520}

\begin{document}
\noindent \textbf{\Large{Caleb Logemann \\
MATH 520 Methods of Applied Math II \\
Homework 9
}}

\pagebreak
\subsection*{Section 14.5}
\begin{enumerate}
  \item[\#5]
    Let $Lu = a_2(x) u'' + a_1(x)u' + a_0(x) u$ with $a_2' = a$, so that $L$ is
    formally self adjoint.
    If $B_1 u = C_1 u(a) + C_2u'(a)$, $B_2 u = C_3u(b) + C_4u'(b)$, show that
    $\set{B_1^*, B_2^*} = \set{B_1, B_2}$.

  \pagbreak
  \item[\#8]
    When we rewrite $a_2(x) u'' + a_1(x) u' + a_0(x) u = \lambda u$ as
    \[
      -(p(x)u')' + q(x)u = \lambda \rho(a) u
    \]
    the latter is often referred to as the \textit{Liouville normal form}.
    Consider the eigenvalue problem
    \begin{align*}
      x^2 u'' + xu' + u = \lambda u \qquad 1 < x < 2 \\
      u(1) = u(2) = 0
    \end{align*}
    \begin{enumerate}
      \item[(a)]
        Find the Liouville normal form.

      \item[(b)]
        What is the orthogonality relationship satisfied by the eigenfunctions?

      \item[(c)]
        Find the eigenvalues and eigenfunctions.
        (You may find the original form of the equation easier to work with than
        the Liouville normal form when computing the eigenvalues and
        eigenfunctions.)
    \end{enumerate}

  \pagbreak
  \item[\#10]
    Consider the Sturm-Liouville problem
    \begin{align*}
      u'' + \lambda u = 0 \qquad 0 < x < 1 \\
      u(0) - u'(0) = u(1) = 0
    \end{align*}
    \begin{enumerate}
      \item[(a)]
        Multiply the equation by $u$ and integrate by parts to show that any
        eigenvalue is positive.

      \item[(b)]
        Show that the eigenvalues are the positive solutions of
        $\tan{\sqrt{\lambda}} = - \sqrt{\lambda}$.

      \item[(c)]
        Show graphically that such roots exist, and form an infinite sequence
        $\lambda_k$ such that $(k - 1/2)\pi < \sqrt{\lambda_k} < k \pi$ and
        \[
          \lim[k \to \infty]{\sqrt{\lambda_k} - (k - 1/2)\pi} = 0
        \]
    \end{enumerate}

  \pagbreak
  \item[\#14]
    If $\set{\psi_n}_{n = 1}^{\infty}$ are Dirichlet eigenfunctions of the
    Laplacian making up an orthonormal basis of $L^2(\Omega)$, let
    $\xi_n = \psi_n/\sqrt{\lambda_n}$
    ($\lambda_n$ the corresponding eigenvalue).
    \begin{enumerate}
      \item[(a)]
        Show that $\set{\xi_n}_{n = 1}^{\infty}$ is an orthonormal basis of
        $H^1_0(\Omega)$.

      \item[(b)]
        Show that $f \in H_0^1(\Omega)$ if and only if
        $\sum{n = 1}{\infty}{\lambda_n \abs{\abr{f, \psi_n}}^2} < \infty$
    \end{enumerate}

  \pagbreak
  \item[\#15]
    If $\Omega < \RR^n$ is a bounded open set with smooth enough boundary, find
    a solution of the wave equation problem
    \begin{align*}
      u_{tt} - \Delta u = 0 \qquad x \in \Omega \quad t > 0 \\
      u(x, t) = 0 \qquad x \in \partial\Omega \quad t > 0 \\
      u(x, 0) = f(x) \quad u_t(x, 0) = g(x) \qquad x \in \Omega
    \end{align*}
    in the form
    \[
      u(x, t) = \sum{n = 1}{\infty}{c_n(t) \psi_n(x)}
    \]
    where $\set{\psi_n}_{n = 1}^{\infty}$ are the Dirichlet eigenfunctions of
    $-\Delta$ in $\Omega$.

  \pagbreak
  \item[\#16]
    Derive formally that
    \[
      G(x, y) = \sum{n=1}{\infty}{\frac{\psi_n(x) \psi_n(y)}{\lambda_n}}
    \]
    where $\lambda_n, \psi_n$ are the Dirichlet eigenvalues and normalized
    eigenfunctions for the domain $\Omega$, and $G(x, y)$ is the corresponding
    Green's function in (14.4.96).
    (Suggestion: if $-\Delta u = f$, expand both $u$ and $f$ in the $\psi_n$
    basis.)

\end{enumerate}
\end{document}
