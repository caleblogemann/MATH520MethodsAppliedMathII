\documentclass[11pt, oneside]{article}
\usepackage[letterpaper, margin=2cm]{geometry}
\usepackage{MATH520}

\begin{document}
\noindent \textbf{\Large{Caleb Logemann \\
MATH 520 Methods of Applied Math II \\
Homework 4
}}

\subsection*{Section 11.4}
\begin{enumerate}
  \item[\#6]
    Let $\v{H} = L^2(0, 1)$ and $T_1u = T_2u = iu'$ with domains
    \begin{align*}
      D(T_1) &= \set{u \in H^1(0, 1): u(0) = u(1)}
      D(T_2) &= \set{u \in H^1(0, 1): u(0) = u(1) = 0}
    \end{align*}
    Show that $T_1$ is self-adjoint, and that $T_2$ is closed and symmetric but
    not self-adjoint.
    What is $T_2^*$?

    \begin{proof}
      
    \end{proof}

  \pagebreak
  \pagebreak
  \item[\#7]
    If $T$ is symmetric with $R(T) = \v{H}$ show that $T$ is self-adjoint.
    % Suggestion: it is enough to show that D(T^*) \subset D(T)

    \begin{proof}
      
    \end{proof}

  \pagebreak
  \item[\#16]
    We say that a linear operator on a Hilbert space $\v{H}$ is bounded below
    if there exists a constant $c_0 > 0$ such that
    \[
      \abr{Tu, u} \ge -c_0 \norm{u}^2 \quad \forall u \in D(T)
    \]
    Show that Theorem 11.6 remains valid if the condition that $T$ be positive
    is replaced by the assumption that $T$ is bounded below.
    % Hint T + c_oI is positive

    \begin{proof}
      
    \end{proof}
\end{enumerate}

\pagebreak
\subsection*{Section 12.4}
\begin{enumerate}
  \item[\#3]
    Recall that the resolvent operator of $T$ is defined to be
    $R_{\lambda} = (\lambda I - T)^{-1}$ for $\lambda \in \rho(T)$.
    \begin{enumerate}
      \item[(a)]
        Prove the resolvant identity (12.1.3).

        \begin{proof}
          
        \end{proof}

      \item[(b)] % Done
        Deduce from this that $R_{\lambda}$ and $R_{\mu}$ commute.

        \begin{proof}
          Let $\lambda, \mu \in \CC$.
          If $\lambda = \mu$, then $R_{\lambda} = R_{\mu}$ so
          \[
            R_{\lambda} R_{\mu} = R_{\lambda}^2 = R_{\mu} R_{\lambda}.
          \]
          In this case $R_{\lambda}$ and $R_{\mu}$ commute trivially.
          Now let $\lambda \neq \mu$, in this case the resolvant identity
          states that
          \[
            R_{\lambda} R_{\mu} = \frac{R_{\lambda} - R_{\mu}}{\lambda - \mu}.
          \]
          Now consider the following
          \begin{align*}
            R_{\lambda} R_{\mu} &= \frac{R_{\lambda} - R_{\mu}}{\lambda - \mu} \\
            &= \frac{R_{\mu} - R_{\lambda}}{\mu - \lambda} \\
            &= R_{\mu} R_{\lambda}
          \end{align*}
          This shows that $R_{\lambda}$ and $R_{\mu}$ commute.
        \end{proof}

      \item[(c)]
        Show also that $T$ and $R_{\lambda}$ commute for $\lambda \in \rho(T)$.

        \begin{proof}
          
        \end{proof}
    \end{enumerate}

  \pagebreak
  \item[\#4]
    Show that $\lambda \to R_{\lambda}$ is continuously differentiable, regarded
    as a mapping from $\rho(T) \subset \CC$ into $\mcB(\v{H})$, with
    \[
      \d{R_{\lambda}}{\lambda} = -R_{\lambda}^2
    \]

    \begin{proof}
      
    \end{proof}
\end{enumerate}
\end{document}
