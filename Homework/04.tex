\documentclass[11pt, oneside]{article}
\usepackage[letterpaper, margin=2cm]{geometry}
\usepackage{MATH520}

\begin{document}
\noindent \textbf{\Large{Caleb Logemann \\
MATH 520 Methods of Applied Math II \\
Homework 4
}}

\subsection*{Section 11.4}
\begin{enumerate}
  \item[\#6]
    Let $\v{H} = L^2(0, 1)$ and $T_1u = T_2u = iu'$ with domains
    \begin{align*}
      D(T_1) &= \set{u \in H^1(0, 1): u(0) = u(1)} \\
      D(T_2) &= \set{u \in H^1(0, 1): u(0) = u(1) = 0}
    \end{align*}
    Show that $T_1$ is self-adjoint, and that $T_2$ is closed and symmetric but
    not self-adjoint.
    What is $T_2^*$?

    \begin{proof}
      First I will show that $T_1$ is self-adjoint.
      Let $v \in D(T_1)$, then $(v, iv')$ is an admissable pair for $T_1^*$.
      To see this note that
      \begin{align*}
        \abr{T_1 u, v} &= \dintt{0}{1}{iu'(x) \overline{v(x)}}{x} \\
        &= \eval{iu(x)\overline{v(x)}}{x = 0}{1} - \dintt{0}{1}{iu(x)\overline{v'(x)}}{x} \\
        &= \eval{iu(x)\overline{v(x)}}{x = 0}{1} + \dintt{0}{1}{u(x)\overline{iv'(x)}}{x} \\
        &= iu(1)\overline{v(1)} - iu(0)\overline{v(0)} + \dintt{0}{1}{u(x)\overline{iv'(x)}}{x}
        \intertext{Since $u(0) = u(1)$ and $v(0) = v(1)$.}
        &= \dintt{0}{1}{u(x)\overline{iv'(x)}}{x} \\
        &= \abr{u, T_1 v}
      \end{align*}
      This shows that $D(T_1) \subset D(T_1^*)$ and that
      $T_1 u = T_1^* u = iu'$ for $u \in D(T)$.

      Now let $v \in D(T_1^*)$, if we can show that $v \in D(T)$ and
      $T_1^* v = iv'$, then we know that $T_1 = T_1^*$.
      Since $v \in D(T_1^*)$ then there exists $g \in L^2(0, 1)$ such that
      $(v, g)$ is an admissable pair for $T_1^*$ and $\dintt{0}{1}{g(x)}{x} = 0$
      Now by definition $T_1^* v = g$.
      Also since $(v, g)$ is an admissable pair,
      \[
        \abr{Tu, v} = \abr{u, g}
      \]
      for all $u \in D(T)$.
      Next I will define the following function
      \[
        G(x) = \dintt{0}{x}{g(y)}{y} + \alpha
      \]
      where
      \[
        \alpha = i\dintt{0}{1}{v(s)}{s} - \dintt{0}{1}{\dintt{0}{s}{g(y)}{y}}{s}.
      \]
      Since $v, g \in L^2(0, 1)$, this function is well-defined.
      Note that by the Fundamental Theorem of Calculus for $L^2$ functions,
      $G'(x) = g(x)$.
      Also note that since $\dintt{0}{1}{g(x)}{x} = 0$,
      \[
        G(0) = \dintt{0}{0}{g(y)}{y} + \alpha = \alpha = \dintt{0}{1}{g(y)}{y} + \alpha = G(1)
      \]
      Now reconsider the inner product $\abr{u, g}$.
      \begin{align*}
        \abr{u, g} &= \dintt{0}{1}{u(x)\overline{g(x)}}{x} \\
        &= \dintt{0}{1}{u(x)\overline{G'(x)}}{x} \\
        &= \eval{u(x)G(x)}{x=0}{1} - \dintt{0}{1}{u'(x)\overline{G(x)}}{x} \\
        &= \p{u(1)G(1) - u(0)G(0)} - \dintt{0}{1}{u'(x)\overline{G(x)}}{x}
        \intertext{Since $u \in D(T)$, $u(0) = u(1)$ and as shown before
          $G(0) = G(1)$, the first term is zero.}
        \abr{u, g} &= -\dintt{0}{1}{u'(x)\overline{G(x)}}{x}
        &= -\abr{u', G}
      \end{align*}
      Now using the definition of admissible pair it is possible to see that
      \[
        \abr{Tu, v} = \abr{u, g} = -\abr{u', G}
      \]
      for all $u \in D(T)$.
      Equivalently this is
      \begin{align*}
        \dintt{0}{1}{iu'(x) \overline{v(x)}}{x} &= -\dintt{0}{1}{u'(x)\overline{G(x)}}{x} \\
        \dintt{0}{1}{u'(x) \overline{G(x) - iv(x)}}{x} &= 0 \\
      \end{align*}
      Since this is true for any $u \in D(T_1)$ this is true in particular for
      \[
        u(x) = \dintt{0}{x}{G(y) - iv(y)}{y}
      \]
      In order to verify that $u \in D(T_1)$, note that $u(x) \in L^2(0, 1)$ and
      that $u'(x) = G(x) - iv(x) \in L^2(0,1)$.
      Also $u(0) = 0$ and
      \begin{align*}
        u(1) &= \dintt{0}{1}{G(y) - iv(y)}{y} \\
        &= \dintt{0}{1}{\dintt{0}{y}{g(s)}{s} + \alpha - iv(y)}{y} \\
        &= \alpha - i\dintt{0}{1}{v(y)}{y} + \dintt{0}{1}{\dintt{0}{y}{g(s)}{s}}{y}
        \intertext{Substituting in for $\alpha$}
        &= i\dintt{0}{1}{v(y)}{y} - \dintt{0}{1}{\dintt{0}{y}{g(s)}{s}}{y} - i\dintt{0}{1}{v(y)}{y} + \dintt{0}{1}{\dintt{0}{y}{g(s)}{s}}{y} \\
        &= 0
      \end{align*}
      Now using this function we see that
      \begin{align*}
        \dintt{0}{1}{\abs{G(x) - iv(x)}^2}{x} &= 0 \\
        \norm{G - iv} &= 0 \\
        G - iv &= 0 \\
        G &= iv \\
      \end{align*}
      Since $G' = g \in L^2(0, 1)$ is differentiable this implies that
      $v' = -iG' \in L^2(0, 1)$.
      Also $v(0) = -iG(0) = -iG(1) = v(1)$ because $G(0) = G(1)$.
      This shows that $v \in D(T_1)$ and that $T_1^* v = g = iv'$.
      Therefore $T_1^* = T_1$.

      Next I will consider $T_2$.
      To see that $T_2$ is symmetric, let $u, v \in D(T^2)$.
      \begin{align*}
        \abr{T_2 u, v} &= \dintt{0}{1}{iu'(x)\overline{v(x)}}{x} \\
        &= \eval{iu(x)\overline{v(x)}}{x=0}{1} - \dintt{0}{1}{iu(x)\overline{v'(x)}}{x} \\
        \intertext{Since $u(0) = u(1) = 0$ and $v(0) = v(1) = 0$}
        &= 0 - \dintt{0}{1}{iu(x)\overline{v'(x)}}{x} \\
        &= \dintt{0}{1}{u(x)\overline{iv'(x)}}{x} \\
        &= \abr{u, T_2 v}
      \end{align*}
      Thus $T_2$ is symmetric.

      To see that $T_2$ is closed let $u_n \in D(T_2)$ such that
      $u_n \to u \in L^2(0, 1)$ and $T_2u_n \to v \in L^2(0, 1)$.
      Since $u_n(0) = u_n(1) = 0$ for all $n$ this implies that
      $u(0) = u(1) = 0$.
      Also since $T_2 u_n \to v$, this implies that $i u_n' \to v$
      Therefore $u_n' \to -iv$ which shows that $u' = -iv$,
      So $u \in D(T_2)$ and $T_2u = iu' = v$ and $T_2$ is closed.
    \end{proof}

  \pagebreak
  \item[\#7] % Done
    If $T$ is symmetric with $R(T) = \v{H}$ show that $T$ is self-adjoint.
    % Suggestion: it is enough to show that D(T^*) \subset D(T)

    \begin{proof}
      Let $T$ be symmetric with $R(T) = \v{H}$.
      Let $u, v \in D(T)$, then
      \[
        \abr{Tu, v} = \abr{u, Tv}.
      \]
      This shows that $(v, Tv)$ is an admissable pair for $T^*$ and that
      $T^* v = Tv$.
      Thus for any $v \in D(T)$, $v \in D(T^*)$, which shows that
      $D(T) \subset D(T^*)$.

      Now let $v \in D(T^*)$, so that there exists $v^* \in \v{H}$ such that
      $(v, v^*)$ is an admissable pair for $T^*$.
      Also since $v^* \in \v{H}$, there exists some $w \in D(T)$ such that
      $Tw = v^*$.
      Now using the definition of admissible pair
      \begin{align*}
        \abr{Tu, v} &= \abr{u, v^*} \\
        &= \abr{u, Tw} \\
        &= \abr{Tu, w}
      \end{align*}
      for all $u \in D(T)$.
      This implies that
      \begin{align*}
        \abr{Tu, v - w} &= 0 \\
        v - w &\perp R(T) = \v{H}\\
        v - w &= 0 \\
        v &= w
      \end{align*}
      This shows that $v \in D(T)$ and that $Tv = v^* = T^* v$.
      Thus $D(T) = D(T^*)$ and $T = T^*$, so $T$ is self-adjoint.
    \end{proof}

  \pagebreak
  \item[\#16] % Done
    We say that a linear operator on a Hilbert space $\v{H}$ is bounded below
    if there exists a constant $c_0 > 0$ such that
    \[
      \abr{Tu, u} \ge -c_0 \norm{u}^2 \quad \forall u \in D(T)
    \]
    Show that Theorem 11.6 remains valid if the condition that $T$ be positive
    is replaced by the assumption that $T$ is bounded below.
    % Hint T + c_oI is positive

    \begin{proof}
      Let $T$ be densely defined, symmetric, and bounded below by $c_0 > 0$.
      Consider the operator $T + c_0 I$, where $I$ is the identity operator on
      $\v{H}$.
      Note that $D(T + c_0 I) = D(T)$ so that $T + c_0 I$ is densely defined.
      Also $T + c_0 I$ is positive.
      To see this let $u \in D(T + c_0 I)$, then
      \begin{align*}
        \abr{(T + c_0 I)u, u} &= \abr{Tu + c_0 u, u} \\
        &= \abr{Tu, u} + c_0 \abr{u, u} \\
        &\ge -c_0 \norm{u}^2 + c_0 \norm{u}^2 \\
        &= 0
      \end{align*}
      The operator $T + c_0 I$ is also symmetric, to show this let
      $u, v \in D(T + c_0 I)$, then
      \begin{align*}
        \abr{(T + c_0 I) u, v} &= \abr{Tu + c_0u, v} \\
        &= \abr{Tu, v} + c_0 \abr{u, v}.
        \intertext{Since $T$ is symmetric}
        \abr{(T + c_0 I) u, v} &= \abr{u, Tv} + c_0 \abr{u, v}.
        \intertext{Also since $c_0 > 0$, $c_0 \in \RR$, so}
        \abr{(T + c_0 I) u, v} &= \abr{u, Tv} + \abr{u, c_0 v} \\
        &= \abr{u, Tv + c_0 v} \\
        &= \abr{u, (T + c_0 I) v}.
      \end{align*}

      Since $T + c_0 I$ is densely defined, positive, and symmetric, by Theorem
      11.6 there exists a positive self-adjoint extension, $S$, of $T + c_0 I$.
      Next I will define the operator $R = S - c_0 I$ and claim that this is
      a bounded below self adjoint extension of $T$.
      First note that $D(R) = D(S) \supset D(T + c_0 I) = D(T)$.
      Let $u \in D(T)$, then
      \begin{align*}
        Ru &= (S - c_0I)u \\
        &= Su - c_0u
        \intertext{Since $u \in D(T + c_0 I)$ and $S$ is an extension of $T + c_0I$}
        Ru &= (T + c_0I)u - c_0u \\
        &= Tu + c_0u - c_0u \\
        &= Tu
      \end{align*}
      This shows that $R$ is an extension of $T$.

      Also $R$ is bounded below, to see this let $u \in D(R)$, then
      \begin{align*}
        \abr{Ru, u} &= \abr{(S - c_0I)u, u} \\
        &= \abr{Su, u} - c_0\abr{u, u}
        \intertext{Since $S$ is positive}
        \abr{Ru, u} &\ge -c_0 \norm{u}^2
      \end{align*}
      Finally I will show that $R$ is self-adjoint.
      Since $S$ and $I$ are self-adjoint.
      \[
        R^* = (S - c_0I)^* = S^* - c_0I^* = S - c_0I = R
      \]
      Therefore for any densely defined, bounded below, symmetric operator
      there exists a bounded below self-adjoint extension.
    \end{proof}
\end{enumerate}

\pagebreak
\subsection*{Section 12.4}
\begin{enumerate}
  \item[\#3] % Done
    Recall that the resolvent operator of $T$ is defined to be
    $R_{\lambda} = (\lambda I - T)^{-1}$ for $\lambda \in \rho(T)$.
    \begin{enumerate}
      \item[(a)] % Done
        Prove the resolvant identity (12.1.3).

        \begin{proof}
          Let $\lambda, \mu \in \rho(T)$.
          Note that $R_{\lambda}^{-1} = \lambda I - T$ and
          $R_{\mu}^{-1} = \mu I - T$ are both defined and
          $R_{\lambda}R_{\lambda}^{-1} = R_{\mu}R_{\mu}^{-1} = I$.
          \begin{align*}
            R_{\lambda} - R_{\mu} &= R_{\lambda}I - IR_{\mu} \\
            &= R_{\lambda}R_{\mu}^{-1}R_{\mu} - R_{\lambda}R_{\lambda}^{-1}R_{\mu} \\
            &= R_{\lambda}\p{R_{\mu}^{-1}R_{\mu} - R_{\lambda}^{-1}R_{\mu}} \\
            &= R_{\lambda}\p{R_{\mu}^{-1} - R_{\lambda}^{-1}}R_{\mu} \\
            &= R_{\lambda}\p{(\mu I - T) - (\lambda I - T)}R_{\mu} \\
            &= R_{\lambda}\p{\mu I - T - \lambda I + T}R_{\mu} \\
            &= R_{\lambda}\p{\mu I - \lambda I}R_{\mu} \\
            &= R_{\lambda}\p{\mu - \lambda}IR_{\mu} \\
            &= \p{\mu - \lambda} R_{\lambda}R_{\mu} \\
          \end{align*}
          This shows that
          \[
            R_{\lambda} - R_{\mu} = \p{\mu - \lambda} R_{\lambda}R_{\mu}
          \]
          for all $\lambda, \mu \in \rho(T)$.
        \end{proof}

      \item[(b)] % Done
        Deduce from this that $R_{\lambda}$ and $R_{\mu}$ commute.

        \begin{proof}
          Let $\lambda, \mu \in \CC$.
          If $\lambda = \mu$, then $R_{\lambda} = R_{\mu}$ so
          \[
            R_{\lambda} R_{\mu} = R_{\lambda}^2 = R_{\mu} R_{\lambda}.
          \]
          In this case $R_{\lambda}$ and $R_{\mu}$ commute trivially.
          Now let $\lambda \neq \mu$, in this case the resolvant identity
          states that
          \[
            R_{\lambda} R_{\mu} = \frac{R_{\lambda} - R_{\mu}}{\lambda - \mu}.
          \]
          Now consider the following
          \begin{align*}
            R_{\lambda} R_{\mu} &= \frac{R_{\lambda} - R_{\mu}}{\lambda - \mu} \\
            &= \frac{R_{\mu} - R_{\lambda}}{\mu - \lambda} \\
            &= R_{\mu} R_{\lambda}
          \end{align*}
          This shows that $R_{\lambda}$ and $R_{\mu}$ commute.
        \end{proof}

      \item[(c)] % Done
        Show also that $T$ and $R_{\lambda}$ commute for $\lambda \in \rho(T)$.

        \begin{proof}
          Let $\lambda \in \rho(T)$, so that $R_{\lambda}^{-1} = \lambda I - T$
          is well-defined.
          \begin{align*}
            TR_{\lambda} &= ITR_{\lambda} \\
            &= R_{\lambda}R_{\lambda}^{-1}TR_{\lambda} \\
            &= R_{\lambda}(\lambda I - T)T R_{\lambda} \\
            &= R_{\lambda}(\lambda T - T^2) R_{\lambda} \\
            &= R_{\lambda} T(\lambda I - T) R_{\lambda} \\
            &= R_{\lambda} T R_{\lambda}^{-1}R_{\lambda} \\
            &= R_{\lambda} T
          \end{align*}
          This shows that $T$ and $R_{\lambda}$ commute for
          $\lambda \in \rho(T)$.
        \end{proof}
    \end{enumerate}

  \pagebreak
  \item[\#4] % Done
    Show that $\lambda \to R_{\lambda}$ is continuously differentiable, regarded
    as a mapping from $\rho(T) \subset \CC$ into $\mcB(\v{H})$, with
    \[
      \d{R_{\lambda}}{\lambda} = -R_{\lambda}^2
    \]

    \begin{proof}
      First I will show that this mapping is continous.
      Let $\lambda_n \in \rho(T)$ such that $\lambda_n \to \lambda$
      as $n \to \infty$.
      Consider
      \begin{align*}
        \lim[n \to infty]{\norm{R_{\lambda_n} - R_{\lambda}}} &= \lim[n \to \infty]{\norm{(\lambda - \lambda_n)R_{\lambda_n} R_{\lambda}}} \\
        &\le \lim[n \to \infty]{\abr{\lambda_n - \lambda}\norm{R_{\lambda_n}}\norm{R_{\lambda}}}
        \intertext{Since $R_{\lambda}$ and $R_{\lambda_n}$ are bounded}
        \lim[n \to infty]{\norm{R_{\lambda_n} - R_{\lambda}}} &= 0
      \end{align*}
      Therefore $R_{\lambda}$ is continuous when seen as a function of $\lambda$.

      Now consider the derivative of $R_{\lambda}$ with respect to $\lambda$.
      \begin{align*}
        \d{R_{\lambda}}{\lambda} &= \lim[n \to \infty]{\frac{R_{\lambda_n} - R_{\lambda}}{\lambda_n - \lambda}} \\
        &= \lim[n \to \infty]{\frac{(\lambda - \lambda_n)R_{\lambda} R_{\lambda_n}}{\lambda_n - \lambda}} \\
        &= \lim[n \to \infty]{-R_{\lambda} R_{\lambda_n}}
        \intertext{Since $R_{\lambda}$ is continuous}
        &= -R_{\lambda}^2
      \end{align*}
      Finally note that $-R_{\lambda}^2$ is continuous as a function of
      $\lambda$ because negation and squaring are continuous operations.
    \end{proof}
\end{enumerate}
\end{document}
