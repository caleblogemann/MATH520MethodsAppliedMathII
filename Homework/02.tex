\documentclass[11pt, oneside]{article}
\usepackage[letterpaper, margin=2cm]{geometry}
\usepackage{MATH520}

\begin{document}
\noindent \textbf{\Large{Caleb Logemann \\
MATH 520 Methods of Applied Math II \\
Homework 2
}}

\subsection*{Section 10.9}
\begin{enumerate}
  \item[\#10] % Done
    Let $S_+$ and $S_-$ be the left and right shift operators on $\mcl^2$.
    Show that $S_- = S_+^*$ and $S_+=S_-^*$.

    \begin{proof}
      Both $S_+$ and $S_-$ are in $\mcB(\mcl^2)$ therefore they both have unique
      adjoints.
      Consider $x, y \in \mcl^2$, then
      \begin{align*}
        \abr{S_+ x, y} &= \sum{n = 1}{\infty}{\p{S_+ x}_n \cdot \overline{y_n}} \\
                       &= \sum{n = 2}{\infty}{x_{n - 1} \cdot \overline{y_n}} \\
                       &= \sum{n = 1}{\infty}{x_{n} \cdot \overline{y_{n+1}}} \\
                       &= \sum{n = 1}{\infty}{x_{n} \cdot \overline{\p{S_- y}_n}} \\
                       &= \abr{x, S_- y}
      \end{align*}
      This shows that $S_+^* = S_-$.
      Now since $S_+, S_- \in \mcB(\mcl^2)$ it is true that $\p{S_+^*}^* = S_+$
      or $S_-^* = S_+$.
    \end{proof}

  \pagebreak
  \item[\#11] % Done
    Let $T$ be the Volterra integral operator $Tu = \dintt{0}{x}{u(y)}{y}$
    considered as an operator on $L^2(0, 1)$.
    Find $T^*$ and $N(T^*)$.

    Consider $u, v \in L^2(0, 1)$.
    \begin{align*}
      \abr{Tu, v} &= \dintt{0}{1}{Tu(x) \overline{v(x)}}{x} \\
                  &= \dintt{0}{1}{\dintt{0}{x}{u(y)}{y} \overline{v(x)}}{x} \\
                  &= \dintt{0}{1}{\dintt{y}{1}{\overline{v(x)}}{x} u(y)}{y} \\
                  &= \dintt{0}{1}{\overline{\dintt{y}{1}{v(x)}{x}} u(y)}{y} \\
                  &= \abr{u, T^* v}
    \end{align*}
    where
    \[
      T^* v(y) = \dintt{y}{1}{v(x)}{x}
    \]
    Since $T \in \mcB(L^2(0, 1))$ this is the unique adjoint of $T$.

    In order to find $N(T^*)$ consider $u \in L^2(0, 1)$ such that
    \[
      T^* u = 0
    \]
    This implies that
    \[
      T^* v(y) = \dintt{y}{1}{v(x)}{x} = 0
    \]
    for every $y \in (0, 1)$.
    Using the Fundamental Theorem of Calculus for $L^2$ functions it can be seen
    that
    \[
      0 = -v(y) + v(1)
    \]
    This implies that $v(y) = v(1)$ for every $y \in (0, 1)$ or equivalently
    that $v$ is a constant function.
    Therefore the
    $N(T^*) = \set{v \in L^2(0, 1): v = c \text{ for some } c \in \RR}$.

  \pagebreak
  \item[\#12] % Done
    Suppose $T \in \mcB(\v{H})$ is self-adjoint and there exists a constant
    $c > 0$ such that $\norm{Tu} \ge c \norm{u}$ for all $u \in \v{H}$.
    Show that there exists a solution of $Tu = f$ for all $f \in \v{H}$.
    Show by example that the conclusion may be false if the assumption of
    self-adjointedness is removed.

    \begin{proof}
      First note that $N(T) = \set{0}$.
      Assume that $u \in N(T)$, then $\norm{u} \le \frac{\norm{Tu}}{c} = 0$.
      Since $\norm{u} = 0$, this implies that $u = 0$.
      Also since $T$ is self-adjoint, $N(T^*) = \set{0}$.
      This implies by Proposition 10.3 that
      $\overline{R(T)} = \v{H} = \set{0}^{\perp}$.
      Therefore the range of $T$ is dense in $\v{H}$, and thus if we can show
      that the range is in fact closed then $Tu = f$ will have a solution for
      all $f \in \v{H}$.

      In order to show that $R(T)$ is closed I will first consider $T^{-1}$.
      Since $N(T) = \set{0}$ this inverse is well defined for all $u \in R(T)$.
      Consider some sequence $u_n \in \R(T)$ such that $u_n \to 0$ as
      $n \to \infty$.
      Now consider the sequence $v_n = T^{-1}u_n$.
      Using the property that $\norm{Tu} \ge c \norm{u}$, it can be seen that
      \[
        \norm{v_n} \le \frac{\norm{Tv_n}}{c} = \frac{\norm{u_n}}{c}
      \]
      Therefore $\lim[n \to \infty]{\norm{v_n}} = 0$ or equivalently
      $T^{-1}u_n \to 0$, this shows that $T^{-1}$ is continous at $0$.
      Also $T^{-1}$ is bounded because $\frac{\norm{T^{-1}u}}{\norm{u}} \le \frac{1}{c}$
      for all $u \in R(T)$.
      Since $T^{-1}$ is bounded and continous at 0, $T^{-1}$ is continous
      everywhere.
      Now it is known that continuity implies that for $F \subset \v{H}$ closed
      $\p{T^{-1}}^{-1}(F)$ is also closed.
      Since $\p{T^{-1}}^{-1} = T$ and $\v{H}$ is closed when considered as a
      subset of itself, this implies that $T(\v{H}) = R(T)$ is closed.
      Now that we have shown that $R(T)$ is closed and dense, this implies
      that $R(T) = \v{H}$.

      This conclusion may be false if that operator is not self-adjoint.
      Consider the operator $S_+$ on $\mcl^2$.
      We have already shown that $S_+^* = S_-$ so $S_+$ is not self-adjoint.
      However $\norm{S_+x} = \norm{x}$ for all $x \in \mcl^2$, so with $c = 1$
      $S_+$ satisfies $\norm{S_+ x} \ge c \norm{x}$ for all $x \in \mcl^2$.
      However $R(S_+) = \set{x \in \mcl^2 : x_1 = 0}$, so $S_+ u = x$ will not
      have a solution if $x_1 \neq 0$.
    \end{proof}

  \pagebreak
  \item[\#13] % Done
    Let $M$ be the multiplication operator $Mu(x) = xu(x)$ in $L^2(0, 1)$.
    Show that $R(M)$ is dense but not closed.

    \begin{proof}
      First of all note that $M$ is self adjoint, this is because
      $M^* u(x) = \overline{x} u(x) = xu(x) = Mu(x)$ on $x \in (0, 1)$.
      Therefore $N(M^*) = N(M)$.
      By definition
      \[
        N(M) = \set{x \in L^2(0, 1): \norm[L^2]{xu(x)} = 0 \forall x \in {0, 1}}
      \]
      Equivalently this implies that
      \[
        \dintt{0}{1}{\abs{xu(x)}^2}{x} = 0
      \]
      which is the same as saying that $u(x) = 0$ almost everywhere.
      This shows that $N(M) = \set{0} = N(M^*)$.
      Now by proposition 10.3, we know that
      $\overline{R(M)} = N(M^*)^{\perp} = L^2(0,1)$.
      Thus the range of $M$ is dense in $L^2(0, 1)$.

      Next I will construct a sequence in $R(M)$ that does not converge
      to a point in $R(M)$.
      That is I will find $u_n \in L^2(0,1)$ such that $Mu_n \to v \not\in R(M)$.
      This will show that $R(M)$ is not closed.
      Let
      \[
        u_n(x) =
        \begin{cases}
          0 & x < \frac{1}{n} \\
          x^{\frac{1}{n} - 1} & x \ge \frac{1}{n}
        \end{cases}
      \]
      First I will show that $u_n(x) \in L^2(0, 1)$ for every $n \in \NN$, $n \ge 3$.
      The fact that $n \ge 3$ is used when taking the antiderivative.
      \begin{align*}
        \dintt{0}{1}{\abs{u_n(x)}^2}{x} &= \dintt{\frac{1}{n}}{1}{\abs{x^{\frac{1}{n} - 1}}^2}{x} \\
        &= \dintt{\frac{1}{n}}{1}{x^{\frac{2}{n} - 2}}{x} \\
        &= \eval{\p{\frac{1}{\frac{2}{n} - 1} x^{\frac{2}{n} - 1}}}{x = \frac{1}{n}}{1} \\
        &= \frac{1}{\frac{2}{n} - 1}\p{1 - \p{\frac{1}{n}}^{\frac{2}{n} - 1}} \\
        &= \frac{n}{2 - n}\p{1 - \p{\frac{1}{n}}^{\frac{2 - n}{n}}} < \infty
      \end{align*}
      Second I will show that $Mu_n(x) \to 1$ in $L^2(0, 1)$.
      Consider
      \begin{align*}
        \norm[L^2(0, 1)]{Mu_n - 1}^2 &= \dintt{0}{1}{\p{Mu_n(x) - 1}^2}{x} \\
        &= \dintt{0}{1}{1 - 2Mu_n(x) + Mu_n(x)^2}{x} \\
        &= \dintt{0}{1}{1}{x} - 2\dintt{0}{1}{Mu_n(x)}{x} + \dintt{0}{1}{Mu_n(x)^2}{x} \\
        &= 1 - 2\dintt{\frac{1}{n}}{1}{x^{\frac{1}{n}}}{x} + \dintt{\frac{1}{n}}{1}{x^{\frac{2}{n}}}{x} \\
        &= 1 - 2\eval{\p{\frac{1}{\frac{1}{n} + 1} x^{\frac{1}{n} + 1}}}{x = \frac{1}{n}}{1} + \eval{\p{\frac{1}{\frac{2}{n} + 1}x^{\frac{2}{n} + 1}}}{x = \frac{1}{n}}{1} \\
        &= 1 - \frac{2}{\frac{1}{n} + 1}\p{1 - \p{\frac{1}{n}}^{\frac{1}{n} + 1}} + \frac{1}{\frac{2}{n} + 1}\p{1 - \p{\frac{1}{n}}^{\frac{2}{n} + 1}}\\
      \end{align*}
      Now consider the limit as $n \to \infty$.
      \begin{align*}
        \lim[n \to \infty]{\norm[L^2(0, 1)]{Mu_n - 1}^2} &= \lim[n \to \infty]{1 - \frac{2}{\frac{1}{n} + 1}\p{1 - \p{\frac{1}{n}}^{\frac{1}{n} + 1}} + \frac{1}{\frac{2}{n} + 1}\p{1 - \p{\frac{1}{n}}^{\frac{2}{n} + 1}}} \\
        &= 1- \lim[n \to \infty]{\frac{2}{\frac{1}{n} + 1}\p{1 - \p{\frac{1}{n}}^{\frac{1}{n} + 1}}} + \lim[n \to \infty]{\frac{1}{\frac{2}{n} + 1}\p{1 - \p{\frac{1}{n}}^{\frac{2}{n} + 1}}} \\
        &= 1 - 2 + 1 = 0
      \end{align*}
      Therefore $Mu_n \to 1$ in $L^2(0, 1)$.
      Lastly I will show that $1 \not\in R(M)$ in order to show that $R(M)$ is
      not closed.
      If $1 \in R(M)$ then there exists $u \in L^2(0, 1)$ such that $xu(x) = 1$.
      This would imply that $u(x) = \frac{1}{x}$, however $\frac{1}{x}$ is not
      in $L^2(0, 1)$.
      To verify this consider the following.
      \begin{align*}
        \dintt{0}{1}{\abs{x^{-1}}^2}{x} &= \dintt{0}{1}{x^{-2}}{x} \\
        &= \eval{-x^{-1}}{x = 0}{1} \\
        &= -1 + \infty = \infty
      \end{align*}
      This shows that $x^{-1}$ is not in $L^2(0, 1)$.
    \end{proof}

  \pagebreak
  \item[\#15] % Done
    An operator $T \in \mcB(\v{H})$ is said to be normal if it commutes with
    its adjoint, i.e. $TT^* = T^*T$.
    Thus, for example, any sefl-adjoint, skey-adjoint, or unitary operator is
    normal.
    For a normal operator $T$ show that
    \begin{enumerate}
      \item[(a)] % Done
        $\norm{Tu} = \norm{T^*u}$ for every $u \in \v{H}$.

        \begin{proof}
          Let $u \in \v{H}$ and consider the following.
          \begin{align*}
            \norm{Tu}^2 &= \abr{Tu, Tu} \\
                        &= \abr{u, T^* T u} \\
                        &= \abr{u, T T^* u} \\
                        &= \abr{u, \p{T^*}^* T^* u} \\
                        &= \abr{T^*u, T^* u} \\
                        &= \norm{T^* u}^2
          \end{align*}
          Therefore $\norm{Tu} = \norm{T^* u}$ for every $u \in \v{H}$.
        \end{proof}

      \item[(b)] % Done
        T is one to one if and only if it has dense range.

        \begin{proof}
          First note that when $T$ is normal $N(T) = N(T^*)$.
          This can be see by using part (a) and by letting $Tu = 0$, then
          $0 = \norm{Tu} = \norm{T^*u}$.
          This implies that $T^* u = 0$.
          Thus if $u \in N(T)$ then $n \in N(T^*)$.
          The opposite direction is equivalent, that is when $u \in N(T^*)$,
          then $u \in N(T)$.

          Now assume that $T$ is one to one or equivalently $N(T) = \set{0}$.
          As was shown earlier this implies that $N(T^*) = \set{0}$ and by
          proposition 10.3 it is clear that
          \[
            \overline{R(T)} = N(T^*)^{\perp} = \set{0}^{\perp} = \v{H}
          \]
          Therefore the range of $T$ is dense in $\v{H}$.

          Finally let $T$ have dense range, that is $\overline{R(T)} = \v{H}$.
          This implies that $N(T^*)^{\perp} = \v{H}$.
          Therefore for all $u \in \v{H}$, $u \perp v$ for all $v \in N(T^*)$.
          The only element of $\v{H}$ that is orthogonal to all of $\v{H}$ is
          the zero element.
          Therefore $N(T^*) = \set{0}$ and it follows that $N(T) = \set{0}$,
          which shows that $T$ is one to one.
        \end{proof}

      \item[(c)] % Done
        Show that any multiplication operator or Fourier multiplication
        operator is normal in $L^2$.

        Let $S$ be a multiplication operator on $L^2$, then $Su(x) = w(x) u(x)$ for
        some $w \in L^{\infty}$.
        We have also shown that $S^* u(x) = \overline{w(x)} u(x)$.
        Now consider any $u \in L^2$.
        \begin{align*}
          S^*Su(x) &= S^*w(x)u(x) \\
          &= \overline{w(x)}w(x) u(x) \\
          &= w(x) \overline{w(x)} u(x) \\
          &= S \overline{w(x)}u(x) \\
          &= SS^* u(x)
        \end{align*}
        Therefore $S^*S = SS^*$ and $S$ is normal.

        Now let $T$ be a Fourier multiplication operator.
        Then $T = F^{-1} S F$ where $F$ is the Fourier Transform and $S$ is some
        multiplication operator.
        Now all of these operators are in $\mcB(L^2)$ so
        $T^* = \p{F^{-1} S F}^* = F^* S^* \p{F^{-1}}^*$.
        Also note that $F$ is unitary so $F^* = F^{-1}$ and $\p{F^{-1}}^* = F$.
        Thus $T^* = F^{-1} S^* F$.
        Now consider
        \begin{align*}
          TT^* &= F^{-1} S F F^{-1} S^* F \\
          &= F^{-1} S S^* F
          \intertext{Since $S$ as a multiplication operator is normal}
          &= F^{-1} S^* S F \\
          &= F^{-1} S^* F F^{-1} S F \\
          &= T^* T
        \end{align*}
        Thus $T$ is a normal operator.

      \item[(b)] % Done
        Show that the shift operators $S_+$ and $S_-$ are not normal in $\mcl^2$.

        Consider $x \in \mcl^2$ such that $x_1 \neq 0$, then $S_- S_+ x = x$ however
        $S_+ S_- x = (0, x_2, x_3, \cdots)$.
        Thus $S_+ S_+^* x = S_+ S_- x \neq S_- S_+ x = S_+^* S_+ x$.
        Therefore $S_+$ is not normal.
        Also $S_- S_-^* x = S_- S_+ x \neq S_+ S_- x = S_-^* S_- x$, so $S_-$ is
        not normal either.
    \end{enumerate}

  \pagebreak
  \item[\#19] % Done
    If $T_n \in \mcB(X)$ and $\sum{n = 1}{\infty}{\norm{T_n}} < \infty$, show
    that the series $\sum{n = 1}{\infty}{T_n}$ is uniformly convergent.

    \begin{proof}
      To show that the series $\sum{n = 1}{\infty}{T_n}$ is uniformly
      convergent is equivalent to showing that the series of partial sums
      converges uniformly to the infinite sum.
      Consider the following
      \begin{align*}
        \norm{\sum{n = 1}{\infty}{T_n} - \sum{n = 1}{N - 1}{T_n}} &= \norm{\sum{n = N}{\infty}{T_n}} \\
        &\le \sum{n = N}{\infty}{\norm{T_n}}
        \intertext{However since this is part of a convergent sum as $N$ goes to infinity this
        sum goes to 0, therefore}
        \lim[N \to \infty]{\norm{\sum{n = 1}{\infty}{T_n} - \sum{n = 1}{N - 1}{T_n}}} = 0
      \end{align*}
      This shows that the series $\sum{n = 1}{\infty}{T_n}$ is uniformly convergent.
    \end{proof}

\end{enumerate}
\end{document}
