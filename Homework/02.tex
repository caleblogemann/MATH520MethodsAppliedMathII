\documentclass[11pt, oneside]{article}
\usepackage[letterpaper, margin=2cm]{geometry}
\usepackage{MATH520}

\begin{document}
\noindent \textbf{\Large{Caleb Logemann \\
MATH 520 Methods of Applied Math II \\
Homework 2
}}

\subsection*{Section 10.9}
\begin{enumerate}
  \item[\#10]
    Let $S_+$ and $S_-$ be the left and right shift operators on $\mcl^2$.
    Show that $S_- = S_+^*$ and $S_+=S_-^*$.

    \begin{proof}
      
    \end{proof}

  \pagebreak
  \item[\#11]
    Let $T$ be the Volterra integral operator $Tu = \dintt{0}{x}{u(y)}{y}$
    considered as an operator on $L^2(0, 1)$.
    Find $T^*$ and $N(T^*)$.

  \pagebreak
  \item[\#12]
    Suppose $T \in \mcB(\v{H})$ is self-adjoint and there exists a constant
    $c > 0$ such that $\norm{Tu} \ge c \norm{u}$ for all $u \in \v{H}$.
    Show that there exists a solution of $Tu = f$ for all $f \in \v{H}$.
    Show by example that the conclusion may be false if the assumption of
    self-adjointedness is removed.

    \begin{proof}
      

      This conclusion may be false if that operator is not self-adjoint.
      Consider the operator $S_+$ on $\mcl^2$.
      We have already shown that $S_+^* = S_-$ so $S_+$ is not self-adjoint.
      However $\norm{S_+x} = \norm{x}$ for all $x \in \mcl^2$, so with $c = 1$
      $S_+$ satisfies $\norm{S_+ x} \ge c \norm{x}$ for all $x \in \mcl^2$.
      However $R(S_+) = \set{x \in \mcl^2 : x_1 = 0}$, so $S_+ u = x$ will not
      have a solution if $x_1 \neq 0$.
    \end{proof}

  \pagebreak
  \item[\#13]
    Let $M$ be the multiplication operator $Mu(x) = xu(x)$ in $L^2(0, 1)$.
    Show that $R(M)$ is dense but not closed.

  \pagebreak
  \item[\#15]
    An operator $T \in \mcB(\v{H})$ is said to be normal if it commutes with
    its adjoint, i.e. $TT^* = T^*T$.

  \pagebreak
  \item[\#19]
\end{enumerate}
\end{document}
