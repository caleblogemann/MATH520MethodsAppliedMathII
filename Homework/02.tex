\documentclass[11pt, oneside]{article}
\usepackage[letterpaper, margin=2cm]{geometry}
\usepackage{MATH520}

\begin{document}
\noindent \textbf{\Large{Caleb Logemann \\
MATH 520 Methods of Applied Math II \\
Homework 2
}}

\subsection*{Section 10.9}
\begin{enumerate}
  \item[\#10] % Done
    Let $S_+$ and $S_-$ be the left and right shift operators on $\mcl^2$.
    Show that $S_- = S_+^*$ and $S_+=S_-^*$.

    \begin{proof}
      Both $S_+$ and $S_-$ are in $\mcB(\mcl^2)$ therefore they both have unique
      adjoints.
      Consider $x, y \in \mcl^2$, then
      \begin{align*}
        \abr{S_+ x, y} &= \sum{n = 1}{\infty}{\p{S_+ x}_n \cdot \overline{y_n}} \\
                       &= \sum{n = 2}{\infty}{x_{n - 1} \cdot \overline{y_n}} \\
                       &= \sum{n = 1}{\infty}{x_{n} \cdot \overline{y_{n+1}}} \\
                       &= \sum{n = 1}{\infty}{x_{n} \cdot \overline{\p{S_- y}_n}} \\
                       &= \abr{x, S_- y}
      \end{align*}
      This shows that $S_+^* = S_-$.
      Now since $S_+, S_- \in \mcB(\mcl^2)$ it is true that $\p{S_+^*}^* = S_+$
      or $S_-^* = S_+$.
    \end{proof}

  \pagebreak
  \item[\#11] % Done
    Let $T$ be the Volterra integral operator $Tu = \dintt{0}{x}{u(y)}{y}$
    considered as an operator on $L^2(0, 1)$.
    Find $T^*$ and $N(T^*)$.

    Consider $u, v \in L^2(0, 1)$.
    \begin{align*}
      \abr{Tu, v} &= \dintt{0}{1}{Tu(x) \overline{v(x)}}{x} \\
                  &= \dintt{0}{1}{\dintt{0}{x}{u(y)}{y} \overline{v(x)}}{x} \\
                  &= \dintt{0}{1}{\dintt{y}{1}{\overline{v(x)}}{x} u(y)}{y} \\
                  &= \dintt{0}{1}{\overline{\dintt{y}{1}{v(x)}{x}} u(y)}{y} \\
                  &= \abr{u, T^* v}
    \end{align*}
    where
    \[
      T^* v(y) = \dintt{y}{1}{v(x)}{x}
    \]
    Since $T \in \mcB(L^2(0, 1))$ this is the unique adjoint of $T$.

    In order to find $N(T^*)$ consider $u \in L^2(0, 1)$ such that
    \[
      T^* u = 0
    \]
    This implies that
    \[
      T^* v(y) = \dintt{y}{1}{v(x)}{x} = 0
    \]
    for every $y \in (0, 1)$.
    Using the Fundamental Theorem of Calculus for $L^2$ functions it can be seen
    that
    \[
      0 &= -v(y) + v(1)
    \]
    This implies that $v(y) = v(1)$ for every $y \in (0, 1)$ or equivalently
    that $v$ is a constant function.
    Therefore the
    $N(T^*) = \set{v \in L^2(0, 1): v = c \text{ for some } c \in \RR}$.

  \pagebreak
  \item[\#12]
    Suppose $T \in \mcB(\v{H})$ is self-adjoint and there exists a constant
    $c > 0$ such that $\norm{Tu} \ge c \norm{u}$ for all $u \in \v{H}$.
    Show that there exists a solution of $Tu = f$ for all $f \in \v{H}$.
    Show by example that the conclusion may be false if the assumption of
    self-adjointedness is removed.

    \begin{proof}


      This conclusion may be false if that operator is not self-adjoint.
      Consider the operator $S_+$ on $\mcl^2$.
      We have already shown that $S_+^* = S_-$ so $S_+$ is not self-adjoint.
      However $\norm{S_+x} = \norm{x}$ for all $x \in \mcl^2$, so with $c = 1$
      $S_+$ satisfies $\norm{S_+ x} \ge c \norm{x}$ for all $x \in \mcl^2$.
      However $R(S_+) = \set{x \in \mcl^2 : x_1 = 0}$, so $S_+ u = x$ will not
      have a solution if $x_1 \neq 0$.
    \end{proof}

  \pagebreak
  \item[\#13]
    Let $M$ be the multiplication operator $Mu(x) = xu(x)$ in $L^2(0, 1)$.
    Show that $R(M)$ is dense but not closed.

  \pagebreak
  \item[\#15]
    An operator $T \in \mcB(\v{H})$ is said to be normal if it commutes with
    its adjoint, i.e. $TT^* = T^*T$.
    Thus, for example, any sefl-adjoint, skey-adjoint, or unitary operator is
    normal.
    For a normal operator $T$ show that
    \begin{enumerate}
      \item[(a)] % Done
        $\norm{Tu} = \norm{T^*u}$ for every $u \in \v{H}$.

        \begin{proof}
          Let $u \in \v{H}$ and consider the following.
          \begin{align*}
            \norm{Tu}^2 &= \abr{Tu, Tu} \\
                        &= \abr{u, T^* T u} \\
                        &= \abr{u, T T^* u} \\
                        &= \abr{u, \p{T^*}^* T^* u} \\
                        &= \abr{T^*u, T^* u} \\
                        &= \norm{T^* u}^2
          \end{align*}
          Therefore $\norm{Tu} = \norm{T^* u}$ for every $u \in \v{H}$.
        \end{proof}

      \item[(b)]
        T is one to one if and only if it has dense range.

        \begin{proof}
          
        \end{proof}

      \item[(c)]
        Show that any multiplication operator or Fourier multiplication
        operator is normal in $L^2$.

      \item[(b)]
        Show that the shift operators $S_+$ and $S_-$ are not normal in $\mcl^2$.

    \end{enumerate}

  \pagebreak
  \item[\#19]
    If $T_n \in \mcB(X)$ and $\sum{n = 1}{\infty}{\norm{T_n}} < \infty$, show
    that the series $\sum{n = 1}{\infty}{T_n}$ is uniformly convergent.

\end{enumerate}
\end{document}
