\documentclass[11pt, oneside]{article}
\usepackage[letterpaper, margin=2cm]{geometry}
\usepackage{MATH520}

\begin{document}
\noindent \textbf{\Large{Caleb Logemann \\
MATH 520 Methods of Applied Math II \\
Homework 3
}}

\subsection*{Section 11.4}
\begin{enumerate}
  \item[\#2] % Done
    Verify that $\v{H} \times \v{H}$ is a Hilbert space with the inner product
    given by (11.1.2), and prove Proposition 11.1.

    \begin{proof}
      First note that the inner product
      \[
        \abr{(u_1, v_1), (u_2, v_2)} = \abr{u_1, u_2} + \abr{v_1, v_2}
      \]
      satisfies all of the axioms of an inner product.

      Next I will show that $\v{H} \times \v{H}$ is complete under this inner
      product.
      Let $x_n = (u_n, v_n)$ be a Cauchy sequence in $\v{H} \times \v{H}$.
      Let $\epsilon > 0$ be given then there exists $N \in \NN$ such that
      $\norm{(u_n, v_n) - (u_m, v_m)} < \epsilon$ for all $n, m > N$.
      This implies that 
      \begin{align*}
        \norm{(u_n, v_n) - (u_m, v_m)}^2 &\le \epsilon^2 \\
        \norm{(u_n - u_m, v_n - v_m)}^2 &\le \epsilon^2 \\
        \abr{(u_n - u_m, v_n - v_m), (u_n - u_m, v_n - v_m)} &\le \epsilon^2 \\
        \abr{u_n - u_m, u_n - u_m} + \abr{v_n - v_m, v_n - v_m} &\le \epsilon^2 \\
        \norm{u_n - u_m}^2 + \norm{v_n - v_m}^2 &\le \epsilon^2
        \intertext{Since the left had is the sum of nonnegative numbers each term
          must be less than $\epsilon^2$}
        \norm{u_n - u_m} &\le \epsilon
        \norm{v_n - v_m} &\le \epsilon
      \end{align*}
      This shows that the sequences $u_n, v_n \in \v{H}$ are both Cauchy.
      Since $\v{H}$ is complete this implies that these sequences converge.
      Let $u = \lim[n \to \infty]{u_n}$ and $v = \lim[n \to \infty]{v_n}$.
      Now consider
      \begin{align*}
        \norm{(u_n, v_n) - (u, v)}^2 &= \norm{(u_n - u, v_n - v}^2 \\
        &= \abr{(u_n - u, v_n - v), (u_n - u, v_n - v)} \\
        &= \abr{u_n - u, u_n - u} + \abr{v_n - v, v_n - v} \\
        &= \norm{u_n - u}^2 + \norm{v_n - v}^2
        \intertext{Now let $n \to \infty$}
        \lim[n \to \infty]{\norm{(u_n, v_n) - (u, v)}^2} &= \lim[n \to \infty]{\norm{u_n - u}^2 + \norm{v_n - v}^2} \\
        &= \lim[n \to \infty]{\norm{u_n - u}^2} + \lim[n \to \infty]{\norm{v_n - v}^2} \\
        \intertext{Since $u_n \to u$ and $v_n \to v$}
        &= 0 + 0 = 0 \\
      \end{align*}
      Thus $(u_n, v_n) \to (u, v)$ as $n \to \infty$.
      This shows that any Cauchy sequence in $\v{H} \times \v{H}$ converges.
      Thus $\v{H} \times \v{H}$ is complete and is also a Hilbert space.

      Finally I will prove Proposition 11.1.
      Let $T:D(T) \subset \v{H} \to \v{H}$ be a closed linear operator.
      Consider the graph of $T$.
      Obviously $G(T)$ is a subspace because $T$ is linear.
      Since $T0 = 0$, this implies that $(0, 0) \in G(T)$.
      Let $(u_1, v_1), (u_2, v_2) \in G(T)$, then $Tu_1 = v_1$, $Tu_2 = v_2$,
      and $T(c_1 u_1 + c_2 u_2) = c_1 v_1 + c_2 v_2$ for any scalars $c_1, c_2$.
      This shows that $c_1(u_1, v_1) + c_2(u_2, v_2) \in G(T)$, and therefore
      $G(T)$ is a subspace.
      Now let $(u_n, v_n) \in G(T)$ be a convergent sequence, that is
      $(u_n, v_n) \to (u, v)$.
      Since $(u_n, v_n) \in G(T)$ this implies that $u_n \in D(T)$ and
      $v_n = Tu_n$.
      Also since $(u_n, v_n) \to (u, v)$ this implies that $u_n \to u$ and
      $v_n = Tu_n \to v$.
      Therefore since $T$ is closed this implies that $u \in D(T)$ and $Tu = v$.
      This shows that $(u, v) \in G(T)$, and thus $G(T)$ is closed.

      Let $T:D(T) \subset \v{H} \to \v{H}$ be a linear operator and let $G(T)$,
      the graph of $T$, be a closed subspace.
      Let $u_n \in D(T)$, such that $u_n \to u$ and $Tu_n \to v$.
      This implies that $(u_n, Tu_n) \in G(T)$ and that $(u_n, Tu_n) \to (u, v)$.
      Since $G(T)$ is closed this implies that $(u, v) \in G(T)$, and thus
      $u \in D(T)$ and $Tu = v$.
      Therefore $T$ is a closed operator.
    \end{proof}

  \pagebreak
  \item[\#5] % Done
    If $T: D(T) \subset \v{H} \to \v{H}$ is a densely defined linear operator,
    $v \in \v{H}$ and the map $u \to \abr{Tu, v}$ is bounded on $D(T)$, show
    that there exists $v^* \in \v{H}$ such that $(v, v^*)$ is an admissible pair
    for $T^*$.

    \begin{proof}
      Let $T:D(T) \subset \v{H} \to \v{H}$ be a densely defined linear
      operator, let $v \in \v{H}$ and let $\phi_v(u) = \abr{Tu, v}$ be a bounded
      functional on $D(T)$.
      Note that since $\phi_v(u) = \abr{Tu, v}$ is bounded it is also continous
      and there exists a continuous extension of $\phi_v$ onto
      $\overline{D(T)}$ by Proposition 10.1.
      Let $S:\overline{D(T)} \to \v{H}$ be this continuous extension.
      Since $T$ is densely defined, $\overline{D(T)} = \v{H}$.
      Therefore $S \in \mcB(\v{H}, \CC)$ so the Riesz Representation Theorem
      states that there exists $v^* \in \v{H}$ such that $Su = \abr{u, v^*}$ for
      all $u \in \v{H}$.
      Now note that for $u \in D(T)$, $Su = \abr{Tu, v}$ and $Su = \abr{u, v^*}$,
      therefore $\abr{Tu, v} = \abr{u, v^*}$.
      This implies that $(v, v^*)$ is an admissable pair for $T^*$.
    \end{proof}

  \pagebreak
  \item[\#8] % Done
    Show that if $T$ is self-adjoint and one-to-one then $T^{-1}$ is also
    self-adjoint.
    % Hint All you really need to do is show that T^{-1} is densely defined.

    \begin{proof}
      Let $T$ be self-adjoint and one-to-one.
      This implies that $T$ is linear and densely defined as the adjoint of $T$
      isn't defined for $T$ not linear and densely defined.
      Also $T = T^*$, and $N(T) = \set{0}$ because $T$ is one-to-one.
      By Theorem 11.3
      \[
        \overline{R(T)} = N(T^*)^{\perp} = N(T)^{\perp} = \set{0}^{\perp} = \v{H}
      \]
      This shows that the range of $T$ is dense in $\v{H}$.
      Since $D(T^{-1}) = R(T)$ this also shows that $T^{-1}$ is densely defined
      on $\v{H}$.
      Now since the range of $T$ is dense, Proposition 11.6 implies that
      $T^*$ is one-to-one and $\p{T^*}^{-1} = \p{T^{-1}}^*$.
      However since $T$ is self-adjoint this is equivalent to
      $T^{-1} = \p{T^{-1}}^*$, which states that $T^{-1}$ is self-adjoint.
    \end{proof}

  \pagebreak
  \item[\#11]
    Assume that $T$ is closed and $S$ is bounded
    \begin{enumerate}
      \item[(a)]
        Show that $S + T$ is closed

        \begin{proof}
          First note that since $S$ is bounded there exists a continuous
          extension of $S$ onto the closure of its domain, by Proposition 10.1.
          Also we can assume that $S$ has been replaced by this extension, and
          therefore $D(S)$ is closed.
          Now let $u_n \in D(S + T)$ and let $u_n \to u$ and $(S + T)u_n \to v$.
          This implies that $u_n \in D(S) \cap D(T)$ and that $Su_n + Tu_n \to v$.
          Since $u_n \in D(S)$ and $D(S)$ is closed, $u \in D(S)$.
          Also since $S$ is bounded it is continuous, therefore $Su_n \to Su$.
          Note that
          \[
            \lim[n \to \infty]{Su_n + Tu_n} = Su + \lim[n \to \infty]{Tu_n} = v
          \]
          Therefore $Tu_n \to v - Su$.
          Also since $u_n \in D(T)$ and $u_n \to u$ by the closedness of $T$
          this implies that $u \in D(T)$ and $Tu = v - Su$.
          Therefore $u \in D(S + T)$ and $Su + Tu = (S + T)u = v$, so $S + T$
          is closed.
        \end{proof}

      \item[(b)]
        Show that $TS$ is closed, but that $ST$ is not closed, in general.

        \begin{proof}
          As in part (a) it can be assumed that $S$ is replaced with its unique
          continuous extension to $\overline{D(S)}$.
          Let $u_n \in D(TS)$ such that $u_n \to u$ and $TSu_n \to v$.
          Note that the domain of $TS$ can be expressed as follows.
          \[
            D(TS) = \set{x \in D(S): Sx \in D(T)}
          \]
          This implies that $u_n \in D(S)$ and since $D(S)$ is closed $u \in D(S)$.
          Therefore because $S$ is bounded and continous $Su_n \to Su$.
          Now note that since $Su_n \in D(T)$ and $Su_n \to Su$ and
          $T(Su_n) \to v$, by the closedness of $T$, $Su \in D(T)$ and $TSu = v$.
          Therefore $u \in D(TS)$ and $TSu = v$, so $TS$ is closed.

          However the operator $ST$ is not closed in general.
          Let $\v{H} = L^2(0, 1)$ and consider the closed operator $Tu = u'$ on
          \[
            D(T) = \set{u \in H^1(0, 1): u(0) = 0}
          \]
          and let $S:\v{H} \to \v{H}$ be the zero operator that is
          $Su = 0$ for all $u \in \v{H}$.
          Note that $S$ is clearly bounded.
        \end{proof}
    \end{enumerate}

  \pagebreak
  \item[\#14] % Done
    If $T$ is closable, show that $T$ and $\overline{T}$ have the same adjoint.

    \begin{proof}
      Since $T$ has an adjoint, $T$ must be densely defined.
      By Theorem 11.5 this implies that $T^*$ is densely defined and
      $\overline{T} = T^{**}$.
      Also since $T^*$ is closed it is also closable so applying Theorem 11.5
      to $T^*$ implies that $T^{**}$ is densely defined and $\overline{T^*} = T^{***}$.
      Since $T^*$ is closed $T^* = \overline{T^*}$, so $T^* = T^{***} = \overline{T}^*$.
      This shows that $T$ and $\overline{T}$ have the same adjoint.
    \end{proof}

\end{enumerate}
\end{document}
