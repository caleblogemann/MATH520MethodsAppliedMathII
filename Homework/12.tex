\documentclass[11pt, oneside]{article}
\usepackage[letterpaper, margin=2cm]{geometry}
\usepackage{MATH520}

\begin{document}
\noindent \textbf{\Large{Caleb Logemann \\
MATH 520 Methods of Applied Math II \\
Homework 12
}}

\subsection*{Section 16.8}
\begin{enumerate}
  \item[\#20]
    Show that the Fr\'echet derivative, if it exists, must be unique.

    \begin{proof}
      Let $X, Y$ be Banach spaces and let $F: D(F) \subset X \to Y$.
      Now suppose that $A_1, A_2 \in B(X, Y)$ exist such that $A_1 \neq A_2$ and
      they both are the Fr\'echet derivative of $F$ at some $x_0 \in D(F)$.
      This means that
      \[
        \lim*[x \to x_0]{\frac{\norm{F(x) - F(x_0) - A_1(x - x_0)}}{\norm{x - x_0}}} = 0
      \]
      and
      \[
        \lim*[x \to x_0]{\frac{\norm{F(x) - F(x_0) - A_2(x - x_0)}}{\norm{x - x_0}}} = 0
      \]
    \end{proof}

  \item[\#21]
    If $F: \RR^2 \to \RR$ is defined by
    \[
      F(x, y) =
      \begin{cases}
        \frac{xy^2}{x^2 + y^4} & (x, y) \neq (0, 0) \\
        0 & (x, y) = (0, 0)
      \end{cases}
    \]
    show that $F$ is G\^ateaux differentiable but not Fr\'echet differentiable
    at the origin.

    The G\^ateaux derivative of $F$ at the origin can be computed as follows.
    \begin{align*}
      DF(0, 0)(u, v) &= \eval{\d*{F(0 + tu, 0 + tv)}{t}}{t=0}{} \\
      &= \eval{\d*{\frac{tu (tv)^2}{(tu)^2 + (tv)^4}}{t}}{t = 0}{} \\
      &= \eval{\d*{\frac{t uv^2}{u^2 + t^2v^4}}{t}}{t = 0}{} \\
      &= \eval{\frac{(u^2 + t^2v^4)uv^2 + tuv^2(2v^4 t)}{(u^2 + t^2 v^4)^2}}{t = 0}{} \\
      &= \frac{u^3v^2}{u^4} \\
      &= \frac{v^2}{u}
    \end{align*}
    This shows that the G\^ateaux derivative of $F$ is $A(u, v) = \frac{v^2}{u}$

    Now we will consider the Fr\'echet derivative of $F$ at $(0, 0)$.
    If the Fr\'echet derivative exists, then $A \in B(X, Y)$ will exist such
    that
    \[
      \lim*[x \to x_0]{\frac{\norm{F(x) - F(x_0) - A(x - x_0)}}{\norm{x - x_0}}} = 0.
    \]
    This can be simplified by noting that $x_0 = (0, 0)$ and using the
    definition of $F$.
    \[
      \lim*[(u, v) \to (0, 0)]{\frac{\abs{\frac{uv^2}{u^2 + v^4} - A(u, v)}}{\sqrt{u^2 + v^2}}} = 0.
    \]

  \item[\#27]
    Let $X, Y$ be Banach spaces, $F: D(F) \subset X \to Y$, and let
    $x, x_0 \in D(F)$ be such that $tx + (1 - t)x_0 \in D(F)$ for
    $t \in \br{0, 1}$.
    If
    \[
      M := \sup*[0 \le t \le 1]{\norm{DF(tx + (1 - t)x_0)}}
    \]
    show that
    \[
      \norm{F(x) - F(x_0)} \le M\norm{x - x_0}
    \]
    (Suggestion: justify and use a suitable version of the fundamental theorem
    of calculus.)
\end{enumerate}

\subsection*{Section 17.5}
\begin{enumerate}
  \item[\#2]
    Let $\lambda_1$ be the smallest Dirichlet eigenvalue for $-\Delta$ in
    $\Omega$, assume that $c \in C(\overline{\Omega})$ and $c(x) > -\lambda_1$
    in $\overline{\Omega}$.
    If $f \in L^2(\Omega)$ prove the existence of a solution of
    \[
      -\Delta u + c(x) u = f \quad x \in \Omega \qquad u = 0 \quad \forall x \in \partial \Omega
    \]  

    \begin{proof}
      
    \end{proof}

  \item[\#3]
    Let $\lambda > 0$ and define
    \[
      A[u, v] = \dintt{\Omega}{}{a_{jk}(x)u_{x_k}(x) v_{x_j}(x)}{x} + \lambda \dintt{\Omega}{}{uv}{x}
    \]
    for all $u, v \in H^1(\Omega)$.
    Assume the ellipticity property (17.1.3) and that $a_{jk} \in L^{\infty}(\Omega)$.
    If $f \in L^2(\Omega)$ show that there exists a unique solution of
    \[
      u \in H^1(\Omega) \qquad A[u, v] = \dintt{\Omega}{}{fv}{x} \forall v \in H^1(\Omega).
    \]
    Justify that $u$ may be regarded as the weak solution of
    \[
      -(a_{jk}u_{x_k})_{x_j} + \lambda u = f(x) \quad x \in \Omega \qquad a_{jk}u_{x_k}n_j = 0 \quad x \in \partial \Omega
    \]
    The above boundary condition is said to be of conormal type.

  \item[\#6]
    Let $f$ and $g$ be in $L^2(0, 1)$.
    Use the Lax-Milgram Theorem to prove there is a unique weak solution
    $\set{u, v} \in H^1_0(0, 1)$ to
    \begin{align*}
      -u'' + u + v' &= f \\
      -v'' + v + u' &= g,
    \end{align*}
    where $u(0) = v(0) = 0$ and $u(1) = v(1) = 0$.
    (Hint: Start by defining the bilinear form
    \[
      A[(u, v), (\phi, \psi)] = \dintt{0}{1}{u'\phi' + u\phi + v'\phi + v'\psi' + v\psi + u'\psi}{x}
    \]
    on $H^1_0(0, 1) \times H^1_0(0, 1)$.

\end{enumerate}
\end{document}
